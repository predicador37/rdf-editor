
\chapter{Introducci�n}

<Intro ... >

\section{Motivaci�n y objetivos}

<Ejemplo biblio \cite{eckel03,gamma95}>

<Ejemplo referencia a secci�n \ref{sec:Estructura-de-la-memoria}
y a subsecci�n de otro cap�tulo, que en este caso es un anexo \ref{subsec:<Primera-subsecci=0000F3n-anexo>}.
\textbf{Para que funcionen correctamente las referencias a otros cap�tulos
al exportar a pdf, debe estar abierto el archivo maestro, abrir/editar
el archivo a exportar desde �l y exportar desde el maestro}.>

<Ejemplo referencia a figura \ref{fig:Ejemplo-de-figura}>

<Ejemplo referencia a tabla \ref{tab:Ejemplo-de-tabla}>

<Ejemplo de nomenclatura \nomenclature{POO}{Programaci�n Orientada a Objetos}
(POO)>

\begin{figure}
\begin{centering}
\includegraphics[width=0.5\columnwidth]{figs/logo_uned}
\par\end{centering}
\caption[Ejemplo de figura con un pie largo]{\label{fig:Ejemplo-de-figura}Ejemplo de figura con un pie largo
para mostrar el uso del �tem del men� de Lyx Insertar->T�tulo breve.
La utilidad del t�tulo breve se aprecia en la lista de figuras, es
ah� donde aparece, si no, aparecer�a todo el pie en la lista, tambi�n
es conveniente para t�tulos de cap�tulos secciones y dem�s largo que
ocupen m�s de una l�nea en la lista de �ndice correspondiente.(Para
insertar salto de l�nea en pie o en enumeraci�n Ctrl-Enter)\protect \\
La figura o tabla hay que insertarla dentro de un flotante.\protect \\
Siempre hay que Insertar->Etiqueta de la figura (fig:Ejemplo-de-figura),
tabla etc, para referenciarla desde el texto.}

\end{figure}

\begin{table}
\begin{centering}
\begin{tabular}{|c|c|c|c|}
\hline 
1 & 2 & 3 & 4\tabularnewline
\hline 
\hline 
aaaaaa & aaaaaa & aaaaaa & aaaaaa\tabularnewline
\hline 
aaaaaa & aaaaaa & aaaaaa & aaaaaa\tabularnewline
\hline 
\end{tabular}
\par\end{centering}
\caption{\label{tab:Ejemplo-de-tabla}Ejemplo de tabla}
\end{table}

<Ejemplo de listado de c�digo, ver listado \ref{lis:Codigo-de-edit},
lo mejor para utilizarlo es copiar el recuadro y cambiar el c�digo
contenido, as� conservaremos las opciones que se han establecido,
como: mostrar n�meros de l�nea, quebrar l�neas largas. etc. Para ver
las opciones y poder modificarlas click derecho sobre el listado y
elegir Configuraci�n. Para cambiar el listado contenido copiar del
IDE el trozo de c�digo que se desee y pegarlo dentro del listado con
Ctrl+May+v, en el men� de Lyx Editar->Pegado especial->Texto simple.>

\begin{lstlisting}[caption={C�digo de edit},label={lis:Codigo-de-edit},language=Java,float=p,numbers=left,basicstyle={\ttfamily},breaklines=true,tabsize=4]
    public final boolean edit() {
        
        ArrayList<String> campos = cargaCampos();
        
        Scanner in = new Scanner( System.in );     
	
	/* La variable resp sera true si el usuario acepta la edicion
	 */
        
	boolean resp = false;        
        int numero_total_opciones = campos.size() + opciones_no_campos;
        System.out.println("\n" + toString()); // muestra el registro
        
        editLoop: while (true) {            
            editMensaje();
            System.out.println("Seleccione un numero (1-" + numero_total_opciones + "):  ");
            int opcion;
            if ( in.hasNextInt() ) {
                opcion = in.nextInt();
                in.nextLine();
            } else {
                System.out.println("\nOPCION NO VALIDA. POR FAVOR, INTRODUZCA UN NUMERO.");
                in.nextLine();
                continue;
            }            
            campos = procesaOpcion(campos, opcion, in);
            if(salir) {
                if(aceptar) 
                    resp = true;
                salir = false;
                aceptar = false;
                break editLoop;                   
            }
            
        }
        return resp;
    }
\end{lstlisting}

\section{Estado actual}

\section{Estructura de la memoria\label{sec:Estructura-de-la-memoria}}

La memoria de esta proyecto se estructura en los siguientes cap�tulos:
\begin{enumerate}
\item Introducci�n general y objetivos
\item <a�adir los dem�s cap�tulos>
\item Conclusiones y trabajos futuros
\end{enumerate}
<Comentar los cap�tulos>
