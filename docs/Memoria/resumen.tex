\noindent \begin{center}
\textbf{\Large{}Resumen}
\par\end{center}{\Large \par}
\setlength{\parskip}{\baselineskip}
\thispagestyle{empty}

El presente proyecto tiene como objeto el desarrollo de un sistema de \textit{playground} para \gls{SPARQL} y \gls{RDF} que habr� de servir como herramienta educativa de introducci�n a las tecnolog�as de la Web Sem�ntica. Su prop�sito es facilitar una interfaz de uso sencilla para que usuarios profanos puedan mantener una primera toma de contacto en este tipo de tecnolog�as a un nivel b�sico, permitiendo la realizaci�n de actividades acad�micas sobre manipulaci�n sencilla de grafos o sobre consultas \gls{SPARQL} al conjunto de datos de trabajo y a \textit{endpoints} externos.

Si bien es cierto que la Web Sem�ntica alcanza hoy en d�a un estado de madurez razonablemente avanzado, el ritmo de cambio de las tecnolog�as de \textit{frontend} en \textit{Javascript} (o JS) es, cuanto menos, vertiginoso. Unido a ello, se tiene que la mayor parte de las implementaciones de componentes de la Web Sem�ntica han sido desarrolladas en tecnolog�as de \textit{backend} como \textit{Java} (\textit{Apache Jena}) o \textit{Python} (\textit{rdflib}), no existiendo a�n est�ndares o implementaciones maduras puramente en Javascript.

Se abordan, por tanto, tres retos: la necesidad de conseguir un producto sencillo y f�cilmente utilizable por usuarios no expertos, la dificultad para encontrar componentes maduros que a�nen Web Sem�ntica con \textit{frontend} y la conveniencia de apostar por un \textit{framework} de desarrollo en \textit{Javascript} estable y con una curva de aprendizaje suave.

Para dar soluci�n a estos problemas, se ha optado por desarrollar una \gls{SPA} con \textit{Vue.js} (un \textit{framework} JS que se jacta de aglutinar las mejores caracter�sticas de \textit{Angular} y \textit{React}, sus principales competidores en el mercado) e integrarlo con las implementaciones recomendadas por el grupo de trabajo de bibliotecas \textit{Javascript} \textit{rdfjs} y con una plataforma de consultas para la web flexible y modular.

Dado el alto nivel de incertidumbre encontrado a la hora de implementar las necesidades del proyecto, se ha utilizado un enfoque metodol�gico incremental e iterativo basado en \textit{Extreme Programming} y apoyado sobre tableros Kanban, lo que ha derivado en una mejor organizaci�n y evoluci�n del proyecto.

El resultado final ofrece un producto b�sico de introducci�n a las tecnolog�as de la Web Sem�ntica y permite pensar en un desarrollo del mismo tan ambicioso como las propias necesidades de los equipos docentes, dado el estado de madurez actual del \textit{frontend} web.

\textbf{Palabras clave:} educaci�n,\textit{ Javascript}, Web Sem�ntica, \textit{Vue.js}, \gls{RDF}, \gls{SPARQL}, ontolog�as.