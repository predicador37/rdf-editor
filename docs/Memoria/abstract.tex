\selectlanguage{english}%
\noindent \begin{center}
\textbf{\Large{}Abstract}
\par\end{center}{\Large \par}

\noindent \thispagestyle{empty}

The aim of this project is to develop a playground system for SPARQL and RDF that will serve as an educational tool for introducing Semantic Web Technologies. Its purpose is to provide a user-friendly interface for profane users to have their first contact in this kind of technologies at a beginner level, allowing them to carry out academic activities on simple graph manipulation or SPARQL queries to the work dataset and to external endpoints.

While it is true that the Semantic Web today reaches a reasonably advanced stage of maturity, the rate of change of frontend technologies in \textit{Javascript} (or JS) is, at least, vertiginous. Moreover, most of the Semantic Web components implementations have been developed in backend technologies such as \textit{Java} (\textit{Apache Jena}) or \textit{Python} (\textit{rdflib}), with a lack of existing standards or pure \textit{Javascript} mature implementations yet.

Three challenges are therefore addressed: the need to obtain a product both simple and easy to use by non-expert users, the difficulty of finding mature components combining the Semantic Web and frontend and the convenience of betting on a stable Javascript development framework with a smooth learning curve.

To solve these problems, the choice made is to develop a Single Page Application (SPA) with \textit{Vue.js} (a JS framework that boasts of agglutinating the best features of \textit {Angular} and \textit {React}, its main competitors in the market) and integrate it with the \textit{RDF Javascript libraries working group} recommended implementations and a flexible and modular web query platform. The system is designed to run in a contemporary web browser, being able to be downloaded from a simple web server (this being the only needed infrastructure for distribution).

Given the high uncertainty found when implementing the project needs, an incremental and iterative methodological approach based on \textit {Extreme Programming} and supported on Kanban boards has been used, which has led to a better project organization and evolution.

The final result provides for a Semantic Web technologies introduction basic product and allows to think of a development as ambitious as the own teaching teams needs, given the current maturity state of the web \textit {frontend}.

\selectlanguage{spanish}%

