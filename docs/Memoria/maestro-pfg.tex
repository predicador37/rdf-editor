\documentclass[12pt,english,spanish,noprefix,refpage]{book}
\usepackage{lmodern}
\renewcommand{\sfdefault}{lmss}
\renewcommand{\ttdefault}{lmtt}
\renewcommand{\familydefault}{\sfdefault}
\usepackage[T1]{fontenc}
\usepackage[latin9]{inputenc}
\usepackage{geometry}
\geometry{verbose,tmargin=2cm,bmargin=2cm,lmargin=2.5cm,rmargin=2cm}
\setcounter{secnumdepth}{3}
\setcounter{tocdepth}{3}
\usepackage{calc}
\usepackage{textcomp}
\usepackage{csquotes}
\usepackage{booktabs}
\usepackage{hyperref}
\usepackage[automake,acronym,toc]{glossaries}
\loadglsentries{glosario}
\makeglossaries
\usepackage{pgfgantt}
\usepackage{array}
\usepackage{wrapfig}
\usepackage{tabularx}
\usepackage{graphicx}
\usepackage{titlesec}
\usepackage{setspace}
\usepackage{svg}
\usepackage[authoryear]{natbib}
\usepackage{nomencl}
% the following is useful when we have the old nomencl.sty package
% \providecommand{\printnomenclature}{\printglossary}
% \providecommand{\makenomenclature}{\makeglossary}
\makenomenclature
% \onehalfspacing
% \renewcommand{\baselinestretch}{1.5}
\linespread{1.25}
\definecolor{barblue}{RGB}{153,204,254}
\definecolor{groupblue}{RGB}{51,102,254}
\definecolor{linkred}{RGB}{165,0,33}
\makeatletter
\raggedbottom
\graphicspath{ {./img/} }
\svgpath{ {./img/} }
\title{RDFplay: un playground para formaci�n en tecnolog�as de la Web Sem�ntica}
\author{Miguel Exp�sito Mart�n}
\let\thetitle\@title
\titleformat{\chapter}[display]   
{\normalfont\huge\bfseries}{\chaptertitlename\ \thechapter}{20pt}{\Huge}   
\titlespacing*{\chapter}{0pt}{-10pt}{30pt}
\usepackage{xcolor}
\hypersetup{ % changes clickable url link colors
	colorlinks,
	linkcolor={red!50!black},
	citecolor={blue!50!black},
	urlcolor={blue!80!black}
}

\newenvironment{changemargin}[2]{%
	\begin{list}{}{%
			\setlength{\topsep}{0pt}%
			\setlength{\leftmargin}{#1}%
			\setlength{\rightmargin}{#2}%
			\setlength{\listparindent}{\parindent}%
			\setlength{\itemindent}{\parindent}%
			\setlength{\parsep}{\parskip}%
		}%
		\item[]}{\end{list}}
\renewcommand{\footnotesize}{\scriptsize}
\setlength{\skip\footins}{1cm}
%%%%%%%%%%%%%%%%%%%%%%%%%%%%%% LyX specific LaTeX commands.
%% Because html converters don't know tabularnewline
\providecommand{\tabularnewline}{\\}

%%%%%%%%%%%%%%%%%%%%%%%%%%%%%% User specified LaTeX commands.
\usepackage{algorithmic}
\usepackage{algorithm}
\floatname{algorithm}{Algoritmo}
\usepackage{layout}
\floatname{table}{Tabla}
\usepackage{url}
\usepackage{hyperref}
\usepackage{fancyhdr}
\fancyhf{}
\setlength{\headheight}{15.2pt}
\pagestyle{fancy}
\renewcommand{\chaptermark}[1]{\markboth{#1}{}}
\makeatletter
\let\sv@endpart\@endpart
\def\@endpart{\thispagestyle{empty}\sv@endpart}
\makeatother


\makeatother

\usepackage{babel}
\addto\shorthandsspanish{\spanishdeactivate{~<>}}

\usepackage{listings}
\addto\captionsenglish{\renewcommand{\lstlistingname}{Listing}}
\addto\captionsspanish{\renewcommand{\lstlistingname}{Listado de c�digo}}
\renewcommand{\lstlistingname}{Listado de c�digo}
\lstset{aboveskip=1.5\baselineskip}
\lstset{belowskip=1.5\baselineskip} 

% Define a style the emphasizes the long form:

\newacronymstyle{em-long-short}
{%
	\GlsUseAcrEntryDispStyle{long-short}%
}%
{%
	\GlsUseAcrStyleDefs{long-short}%  
	\renewcommand*{\genacrfullformat}[2]{%
		\emph{\glsentrylong{##1}}##2\space
		(\firstacronymfont{\glsentryshort{##1}})%
	}%
	\renewcommand*{\Genacrfullformat}[2]{%
		\emph{\Glsentrylong{##1}}##2\space
		(\firstacronymfont{\glsentryshort{##1}})%
	}%
	\renewcommand*{\genplacrfullformat}[2]{%
		\emph{\glsentrylongpl{##1}}##2\space
		(\firstacronymfont{\glsentryshortpl{##1}})%
	}%
	\renewcommand*{\Genplacrfullformat}[2]{%
		\emph{\Glsentrylongpl{##1}}##2\space
		(\firstacronymfont{\Glsentryshortpl{##1}})%
	}%
}


\setacronymstyle{em-long-short}



\begin{document}
\begin{titlepage}\thispagestyle{empty}%
\fbox{\begin{minipage}[c][0.99\textheight][t]{0.95\columnwidth}%
\vspace{1cm}
\begin{center}
\includegraphics[width=2cm]{figs/escudo_uned}
\par\end{center}
\bigskip{}
\begin{singlespace}
\begin{center}
{\large{}UNIVERSIDAD NACIONAL DE EDUCACI�N A DISTANCIA}\vspace{0.7cm}
\par\end{center}
\begin{center}
{\large{}ESCUELA T�CNICA SUPERIOR DE INGENIER�A INFORM�TICA}\vspace{0.5cm}
\par\end{center}
\begin{center}
Proyecto de Fin de Grado en Ingenier�a en Tecnolog�as de la Informaci�n {\Large{}\vspace{0.8cm}
}
\par\end{center}{\Large \par}
\end{singlespace}
\begin{center}
\textbf{\Large{}RDFplay: UN PLAYGROUND }\\
\textbf{\Large{}PARA FORMACI�N EN TECNOLOG�AS}\\
\textbf{\Large{}DE LA WEB SEM�NTICA}
\par\end{center}{\Large \par}
\vfill{}
\begin{flushleft}
\qquad{}MIGUEL EXP�SITO MART�N\bigskip{}
\par\end{flushleft}
\begin{flushleft}
\qquad{}Dirigido por: JOS� LUIS FERN�NDEZ VINDEL
\par\end{flushleft}
\begin{flushleft}
\qquad{}Co-dirigido por: RAFAEL MART�NEZ TOM�S
\par\end{flushleft}
\begin{flushleft}
\vspace{1.5cm}
\par\end{flushleft}
\begin{flushleft}
\qquad{}Curso: 2017-2018: 2� Convocatoria
\par\end{flushleft}
\bigskip{}
%
\end{minipage}}

\end{titlepage}
\thispagestyle{empty}

\cleardoublepage{}

\thispagestyle{empty}%
\fbox{\begin{minipage}[c][0.99\textheight][t]{0.95\columnwidth}%
\begin{center}
\vspace{1cm}
\par\end{center}
\begin{center}
\includegraphics[width=2cm]{figs/escudo_uned}
\par\end{center}
%\bigskip{}
\begin{singlespace}
\begin{center}
\textbf{\Large{}RDFplay: UN PLAYGROUND }\\
\textbf{\Large{}PARA FORMACI�N EN TECNOLOG�AS}\\
\textbf{\Large{}DE LA WEB SEM�NTICA}
\par\end{center}{\large \par}
\end{singlespace}
\begin{center}
Proyecto de Fin de Grado de modalidad oferta espec�fica
\par\end{center}
\begin{flushleft}
\vspace{0.3cm}
\par\end{flushleft}
\begin{flushleft}
\enskip{}Realizado por:{\large{}\hspace{0.5em}}Miguel Exp�sito Mart�n{\large{}}
\vspace{0.3cm}
\par\end{flushleft}
\begin{flushleft}
\enskip{}Dirigido por:{\large{}\hspace{0.5em}}Jos� Luis Fern�ndez Vindel
\par\end{flushleft}
\begin{flushleft}
\enskip{}Co-dirigido por:{\large{}\hspace{0.5em}}Rafael Mart�nez Tom�s
\par\end{flushleft}
\begin{flushleft}
{\large{}}\vspace{0.6cm}
\par\end{flushleft}
\begin{flushleft}
\enskip{}Tribunal calificador
\par\end{flushleft}
\begin{flushleft}
\enskip{}Presidente: D/D�. 
\par\end{flushleft}
\begin{flushleft}
\textbf{\medskip{}
}
\par\end{flushleft}
\noindent \begin{flushleft}
\enskip{}Secretario: D/D�. 
\par\end{flushleft}
\begin{flushleft}
\textbf{\medskip{}
}
\par\end{flushleft}
\noindent \begin{flushleft}
\enskip{}Vocal: D/D�. 
\par\end{flushleft}
\vfill{}
\begin{flushleft}
\enskip{}Fecha de lectura y defensa: 2/10/2018
\par\end{flushleft}
\begin{flushleft}
\enskip{}Calificaci�n: 
\par\end{flushleft}
\smallskip{}
%
\end{minipage}}
\thispagestyle{empty}


\chapter*{Agradecimientos}
\begin{center}
\thispagestyle{empty}
\par\end{center}

A mi familia, a mis amigos, a mi Director de proyecto y a los que, sin querer, me animaron a retomar mis estudios.


\thispagestyle{empty}

\cleardoublepage{}

\pagenumbering{roman}\fancyhead[LE,RO]{\thepage}\fancyhead[LO]{\nouppercase{\leftmark}}\fancyhead[RE]{\nouppercase{\rightmark}}\noindent \begin{center}
\textbf{\Large{}Resumen}
\par\end{center}{\Large \par}
\setlength{\parskip}{\baselineskip}
\thispagestyle{empty}

El presente proyecto tiene como objeto el desarrollo de un sistema de \textit{playground} para SPARQL y RDF que habr� de servir como herramienta educativa de introducci�n a las tecnolog�as de la Web Sem�ntica. Su prop�sito es facilitar una interfaz de uso sencilla para que usuarios profanos puedan introducirse en este tipo de tecnolog�as a un nivel b�sico, permitiendo la realizaci�n de actividades acad�micas sobre manejo sencillo de grafos as� como sobre consultas SPARQL al conjunto de datos de trabajo o a \textit{endpoints} externos.

Si bien es cierto que la Web Sem�ntica alcanza hoy en d�a un estado de madurez razonablemente avanzado, el ritmo de cambio de las tecnolog�as de \textit{frontend} en \textit{Javascript} (en adelante, JS) es, cuanto menos, vertiginoso. Unido a ello, se tiene que la mayor parte de las implementaciones de componentes de la Web Sem�ntica han sido desarrolladas en tecnolog�as de \textit{backend} como \textit{Java} (\textit{Apache Jena}) o \textit{Python} (\textit{rdflib}), no existiendo a�n est�ndares o implementaciones maduras puramente en Javascript.

Se abordan, por tanto, tres retos: la necesidad de conseguir un producto sencillo y f�cilmente utilizable por usuarios no expertos, la dificultad para encontrar componentes maduros que a�nen Web Sem�ntica con \textit{frontend} y la conveniencia de apostar por un \textit{framework} de desarrollo en \textit{Javascript} estable y con una curva de aprendizaje suave.

Para dar soluci�n a estos problemas, se ha optado por desarrollar una  \textit{Single Page Application} (SPA) con Vue.js (un \textit{framework} JS que se jacta de aglutinar las mejores caracter�sticas de Angular y React, sus principales competidores en el mercado) e integrarlo con las implementaciones recomendadas por el grupo de trabajo de bibliotecas \textit{Javascript} \textit{rdfjs} y con una plataforma de consultas para la web flexible y modular. El sistema est� planteado para ejecutarse en un sencillo navegador contempor�neo, descarg�ndose a trav�s de un simple servidor web (siendo este la �nica infraestructura necesaria para su distribuci�n).

Dado el alto nivel de incertidumbre existente a la hora de implementar las necesidades del proyecto, se ha optado por utilizar un enfoque metodol�gico incremental e iterativo basado en \textit{Extreme Programming} y apoyado sobre tableros Kanban, lo que ha permitido una mejor organizaci�n y evoluci�n del proyecto.

El resultado final ofrece un producto b�sico de introducci�n a las tecnolog�as de la Web Sem�ntica y permite pensar en un desarrollo del mismo tan ambicioso como las propias necesidades de los equipos docentes, dado el estado de madurez actual del \textit{frontend} web.\thispagestyle{empty}

\selectlanguage{english}%
\noindent \begin{center}
	\textbf{\Large{}RDFplay: a playground for education on the Semantic Web technologies}
	\par\end{center}{\Large \par}

\noindent \begin{center}
\textbf{\Large{}Abstract}
\par\end{center}{\Large \par}


The aim of this project is to develop a playground system for \gls{SPARQL} and \gls{RDF} that will serve as an educational tool for introducing Semantic Web Technologies. Its purpose is to provide a user-friendly interface for profane users to have their first contact in this kind of technologies at a beginner level, allowing them to carry out academic activities on simple graph manipulation or \gls{SPARQL} queries to the work dataset and to external endpoints.

While it is true that the Semantic Web today reaches a reasonably advanced stage of maturity, the rate of change of frontend technologies in \textit{Javascript} (or JS) is, at least, vertiginous. Moreover, most of the Semantic Web components implementations have been developed in backend technologies such as \textit{Java} (\textit{Apache Jena}) or \textit{Python} (\textit{rdflib}), with a lack of existing standards or pure \textit{Javascript} mature implementations yet.

Three challenges are therefore addressed: the need to obtain a product both simple and easy to use by non-expert users, the difficulty of finding mature components combining the Semantic Web and frontend and the convenience of betting on a stable \textit{Javascript} development framework with a smooth learning curve.

To solve these problems, the choice made is to develop a \gls{SPA} with \textit{Vue.js} (a JS framework that boasts of agglutinating the best features of \textit {Angular} and \textit {React}, its main competitors in the market) and integrate it with the \textit{RDF Javascript libraries working group} recommended implementations and a flexible and modular web query platform. 

Given the high uncertainty found when implementing the project needs, an incremental and iterative methodological approach based on \textit {Extreme Programming} and supported on Kanban boards has been used, which has led to a better project organization and evolution.

The final result provides for a Semantic Web technologies introduction basic product and allows to think of a development as ambitious as the own teaching teams needs, given the current maturity state of the web \textit {frontend}.

\textbf{Keywords:} education, \textit{Javascript}, Semantic Web, \textit{Vue.js}, \gls{RDF}, \gls{SPARQL}, ontologies.

\selectlanguage{spanish}%

\thispagestyle{empty}

\cleardoublepage{}

\tableofcontents{}\cleardoublepage{}\markboth{\nomname}{\nomname}

% \printnomenclature{}

\listoffigures

\renewcommand{\listtablename}{�ndice de tablas} 

\listoftables

\cleardoublepage{}
\fancyhf{}
\pagenumbering{arabic}
\pagestyle{fancy}
\fancyhead[LE,RO]{\thepage}
\fancyhead[LO]{\chaptername {} \thechapter. \leftmark}
\fancyhead[RE]{\slshape RDFplay: un playground para formaci�n en tecnolog�as de la Web Sem�ntica}

\chapter{Introducci�n}

No cabe duda que uno de los retos a los que se enfrenta la Sociedad de la Informaci�n y del Conocimiento es la educaci�n, una de las mejores inversiones en el futuro de Europa. Seg�n la �ltima Comunicaci�n de la Comisi�n al Parlamento, al Consejo, al Comit� Econ�mico y Social Europeo y al Comit� de las Regiones\cite{comission01},  el uso de las TIC para prop�sitos educativos se queda atr�s, tanto en adopci�n como en competencias del profesorado. La innovaci�n en sistemas educativos, entendida como la interiorizaci�n de nuevos servicios y herramientas por parte de las organizaciones educativas, puede mejorar los resultados formativos y la eficiencia.

Siendo una de las prioridades del Plan de Acci�n propuesto por la Comisi�n hacer un mejor uso de la tecnolog�a digital para el aprendizaje y la ense�anza, parece apropiado que instituciones como la Universidad Nacional de Educaci�n a Distancia (UNED) en Espa�a y sus departamentos, como el de Inteligencia Artificial, se propongan aplicar los �ltimos avances tecnol�gicos para mejorar la calidad del sistema educativo y facilitar a los alumnos la adquisici�n de nuevas competencias digitales complejas.

Dentro de estas competencias se pueden encuadrar t�rminos como Web de Datos, Web Sem�ntica o Datos Enlazados, todos definidos por el padre de la Web, Tim Berners-Lee\cite{bernerslee01, bernerslee02}. No son m�s que la evoluci�n l�gica de la Web a la que todos estamos acostumbrados, pero entendida en su crecimiento como una soluci�n a los problemas de la ingente cantidad de datos no estructurados o inconexos que se genera al d�a en 2018 (seg�n Domo\cite{domo01}, 2,5 quintillones de bytes).
 
En concreto, este trabajo pretende acercar las tecnolog�as de la Web Sem�ntica al alumnado siguiendo las recomendaciones del  W3C RDFJS Community Group\cite{rdfw3ccg} a trav�s de una sencilla aplicaci�n puramente desarrollada en \textit{Javascript} y lista para usar, que permita la exploraci�n de facilidades de modelado y consulta sobre un conjunto de datos predefinido o aportado por el propio alumno as� como la carga y realizaci�n de actividades simples propuestas por el profesorado.


La memoria de este proyecto se estructura en los siguientes cap�tulos:

\begin{itemize}
	\item \textbf{Cap�tulo 1. Introducci�n.}  Esta introducci�n.
	\item \textbf{Cap�tulo 2. La Web Sem�ntica.} Presentaci�n de las tecnolog�as que conforman la Web Sem�ntica: datos enlazados, RDF y SPARQL.
	\item \textbf{Cap�tulo 3. Motivaci�n y objetivos.} Exposici�n de las razones que llevaron a la elecci�n de este proyecto en concreto, as� como las metas que se esperaban alcanzar.
	\item \textbf{Cap�tulo 4. Trabajos previos, estado actual y recursos.} Un estudio sobre el estado actual de las tecnolog�as en cuesti�n y los recursos que se estiman necesarios para llevar a cabo con �xito el proyecto.
	\item \textbf{Cap�tulo 5. Metodolog�a y planificaci�n.} Descripci�n de la metodolog�a de desarrollo de software a utilizar y an�lisis de las planificaciones llevada a cabo (inicial y efectiva).
	\item \textbf{Cap�tulo 6. An�lisis.} Presentaci�n un estudio de necesidades para el sistema de informaci�n concretadas en casos de uso detallados.
	\item \textbf{Cap�tulo 7. Dise�o.} Descripci�n de las distintas arquitecturas del sistema, as� como su estructura basada en componentes.
	\item \textbf{Cap�tulo 8. Implementaci�n y pruebas.} Detalles de implementaci�n en Javascript y de las pruebas unitarias y extremo a extremo llevadas a cabo.
	\item \textbf{Cap�tulo 9. Resultados.}
	\item \textbf{Cap�tulo 10. Conclusiones y l�neas futuras de trabajo.}
\end{itemize}\cleardoublepage{}

\chapter{La Web Sem�ntica}

Cabe comenzar el desarrollo de la presente memoria con una breve introducci�n a los fundamentos de la Web Sem�ntica y las tecnolog�as que la sustentan: \gls{RDF} y \gls{SPARQL}. Asimismo, queda definido tambi�n el t�rmino de datos enlazados o \textit{linked data}, estrechamente relacionado con estos conceptos y de gran importancia hoy en d�a en la difusi�n de datos estructurados a trav�s de Internet.

\section{Fundamentos}

La Web Sem�ntica es un t�rmino originalmente acu�ado por Tim Berners-Lee en el a�o 2001\cite{bernerslee01}. Hasta la fecha, la \textit{World Wide Web} hab�a sido concebida como una idea de colaboraci�n abierta en la que m�ltiples contribuciones de varios autores pod�an tener cabida y ser compartidas universalmente. Dichas contribuciones, realizadas en forma de documentos, estaban dirigidas a personas y no a m�quinas o computadores. Berners-Lee vio m�s all� y propuso su extensi�n para lograr su manipulaci�n autom�tica; en resumidas cuentas, permitir que agentes inteligentes (programas de ordenador) fueran capaces de encontrar datos y su significado a trav�s de hiperenlaces a definiciones de t�rminos clave y reglas de razonamiento e inferencia l�gica.

Para lograr tan ambicioso objetivo, los agentes inteligentes necesitar�an tener acceso a contenido y conocimiento estructurado, as� como a las reglas de inferencia necesarias. En este contexto, la Web Sem�ntica pod�a apoyarse en las siguientes tecnolog�as ya existentes:

\begin{minipage}{\linewidth}
\begin{itemize}  
	\item  \textbf{\gls{RDF}}, que permite expresar conocimiento en forma de tripletas o ternas (a modo de sujeto, verbo y objeto en una oraci�n) utilizando URIs (\textit{Uniform Resource Indicators}\footnote{Cadenas de caracteres que identifican los recursos de una red de forma un�voca.}).
	\item \textbf{\gls{XML}}, un metalenguaje que permite crear lenguajes de etiquetado o marcado y documentos a su vez convenientemente etiquetados y con estructura arbitraria.
\end{itemize}
\end{minipage}

Sin embargo, un tercer pilar era necesario para resolver la problem�tica de la existencia de distintos identificadores en distintas bases de datos relacionadas con el mismo significado conceptual. Gracias a las \textbf{ontolog�as}, documentos que definen formalmente las relaciones entre distintos t�rminos, estas diferencias pod�an ser salvadas bien mediante el uso de t�rminos estandarizados bien mediante la definici�n de relaciones conceptuales de igualdad o similitud entre dichos t�rminos.

En palabras del propio Berners-Lee:\textit{ \blockquote[{\cite{bernerslee01}}]{Con un dise�o adecuado, la Web Sem�ntica puede ayudar en la evoluci�n del conocimiento humano como un todo.}}

\section{Linked Data}

Estrechamente relacionado con la Web Sem�ntica se encuentra el concepto de datos enlazados o \textit{Linked Data}, tambi�n propuesto por Berners-Lee en 2006\cite{bernerslee02} en la que se considera como su introducci�n oficial y formal. Se trata de un movimiento respaldado por el propio \gls{W3C}\footnote{\url{https://www.w3c.es/}} que se centra en conectar conjuntos de datos a lo largo de la Web y que puede verse como un subconjunto de la Web Sem�ntica.

Se basa, por tanto, en dos aspectos clave:

\begin{itemize}
	\item La publicaci�n de conjuntos datos estructurados en l�nea.
	\item El establecimiento de enlaces entre dichos conjuntos de datos.
\end{itemize}

La Web Sem�ntica es el fin y los datos enlazados proporcionan el medio para alcanzar dicho fin.

\section{\gls{RDF}}

Las tecnolog�as de la Web Sem�ntica permiten, por tanto, almacenar conocimiento en forma de conceptos y relaciones a trav�s de ternas o tripletas, modelar dominios de conocimiento con vocabularios est�ndar y ofrecer potentes facilidades de consulta sobre dichos modelos. Dentro de estas tecnolog�as, \gls{RDF} es, sin duda, el bloque de construcci�n fundamental (como \gls{HTML} lo ha sido para la Web).

\gls{RDF} fue propuesto por el W3C en 1999\cite{Lassila:99:RDF} como un est�ndar para crear y procesar metadatos con el objetivo de describir recursos independientemente de su dominio de aplicaci�n, ayudando, por tanto, a promover la interoperabilidad entre aplicaciones.

Sin embargo, con la llegada de la Web Sem�ntica, el alcance de \gls{RDF} creci� y ya no s�lo se usa para codificar metadatos sobre recursos web, sino tambi�n \textbf{para describir cualquier recurso as� como sus relaciones} en el mundo real.

El nuevo alcance de \gls{RDF} qued� reflejado en las especificaciones publicadas en 2004 por el \textit{\gls{RDF} Core Working Group}\footnote{Recomendaciones del \gls{W3C}, \url{https://www.w3.org/2001/sw/RDFCore/}}, ahora actualizadas a fecha de 2014\footnote{Versiones actualizadas de las recomendaciones del \gls{W3C}, \url{https://www.w3.org/2011/rdf-wg/wiki/Main\_Page}}, de las que se pueden extraer las siguientes definiciones de \gls{RDF}:


\vspace*{\baselineskip}

	\begin{table}[htb]
		\centering
		\begin{tabular}{p{3cm}p{12cm}}
			\toprule
			Recomendaci�n & Definici�n \\ \midrule
			\textit{\gls{RDF} Primer} & Un marco de trabajo para expresar informaci�n sobre recursos. \\ \midrule
			\textit{\gls{RDF} Concepts \& Syntax} & Un marco de trabajo para representar informaci�n en la Web. \\ 
			\bottomrule
		\end{tabular}
		\caption{Definiciones de \gls{RDF} en las recomendaciones del \gls{W3C}.}
		\label{tabdefrdf}
	\end{table}


\subsection{Modelo Abstracto de \gls{RDF}}

\gls{RDF} ofrece un modelo abstracto que permite descomponer el conocimiento en peque�as piezas llamadas sentencias (\textit{statements}), tripletas o ternas y que toman la forma:

\begin{center}
Sujeto - Predicado - Objeto
\end{center}

En donde \textit{Sujeto} y \textit{Objeto} representan dos conceptos o ``cosas'' en el mundo (tambi�n denominados \textbf{recursos}) y \textit{Predicado} la relaci�n que los conecta.

Los nombres de estos recursos (as� como de los predicados) han de ser globales y deber�an identificarse un�vocamente por un \gls{URI}.

Por tanto, un modelo \gls{RDF} puede expresarse como una colecci�n de tripletas o un grafo, dado que aquellas no son m�s que grafos dirigidos. Los sujetos y objetos ser�an los nodos del grafo y los predicados sus aristas.

Otra nomenclatura utilizada para representar una terna ser�a la siguiente:

\begin{center}
	Recurso - Propiedad - Valor de la propiedad
\end{center}

En donde el valor de la propiedad u \textit{Objeto} tambi�n puede ser un literal, que consiste simplemente en un dato textual bruto.

Cabe destacar que una terna \gls{RDF} s�lo puede modelar relaciones binarias. Para modelar una relaci�n n-aria, suelen utilizarse recursos intermedios como nodos en blanco, que no son m�s que sujetos u objetos que no tienen un \gls{URI} como identificador (nodos sin nombre o an�nimos).

En resumen:

\vspace*{\baselineskip}
	\begin{table}[htb]
		\centering
		\begin{tabular}{p{6cm}p{6cm}}
			\toprule
			Sujeto & Puede ser un \gls{URI} o un nodo en blanco. \\ \midrule
			Predicado o Propiedad & Debe ser un \gls{URI}. \\ \midrule
			Objeto o Valor de la Propiedad & Puede ser un \gls{URI}, un literal o un nodo en blanco. \\
			\bottomrule
		\end{tabular}
		\caption{Resumen conceptual de ternas.}
		\label{tabresterna}
	\end{table}

A continuaci�n se muestran dos ejemplos de tripletas:

\vspace*{\baselineskip}

	\begin{table}[htb]
		\tiny
		\centering
		\begin{tabular}{p{5cm}p{5cm}p{5cm}}
			\toprule
			Sujeto & Predicado & Objeto \\ \midrule
			http://www.uned.es/ia/example\#Analista & http://www.w3.org/1999/02/22-rdf-syntax-ns\#type & http://www.w3.org/2002/07/owl\#Class  \\ \midrule
		    http://www.uned.es/ia/example\#Analista & http://www.w3.org/2000/01/rdf-schema\#label & ``Analista Inform�tico'' \\ 
			\bottomrule
		\end{tabular}
		\caption{Ejemplos de tripletas reales.}
		\label{tabejternas}
	\end{table}


\subsection{Aplicaciones pr�cticas de RDF}

Gracias a \gls{RDF}, es posible construir la Web Sem�ntica y enlazar conjuntos de datos. A continuaci�n se enumeran algunas de sus aplicaciones reales:

\begin{itemize}  
	\item A�adir informaci�n legible a los motores de b�squeda.
	\item Enriquecer un conjunto de datos enlaz�ndolo con conjuntos de datos de terceros.
	\item Facilitar el descubrimiento de \gls{API}s, en este caso referido a puntos accesibles a trav�s de la Web a trav�s de su entrelazado.
	\item Construir agregaciones de datos sobre determinados temas.
	\item Proporcionar un est�ndar de intercambio de datos entre bases de datos.
	\item Enriquecer, describir y contextualizar los datos mediante un enfoque rico en metadatos.
\end{itemize}

Su principal ventaja frente a los modelos y bases de datos relacionales es una mayor facilidad para evolucionar los grafos, algo que en un sistema relacional es complejo (dado que requiere de modificaciones en las estructuras de almacenamiento de datos y sus consultas asociadas). En otras palabras, su \textbf{flexibilidad para modelar lo inesperado}.

Un ejemplo claro a nivel global es \textit{Wikidata}\footnote{\url{https://www.wikidata.org/wiki/Wikidata:Main\_Page}}, una base de datos de conocimiento abierta que puede ser le�da y editada tanto por personas como por m�quinas.\textit{ Wikidata} act�a como un almac�n centralizado para datos estructurados de otros proyectos hermanos como \textit{Wikipedia}, \textit{Wikisource}, etc. En el momento de redacci�n de esta memoria, cuenta con  50.035.140 elementos de datos que cualquiera puede editar.
 
De una forma m�s concreta, es destacable el importante rol de \gls{RDF} y las tecnolog�as de la Web Sem�ntica en �mbitos de \textbf{b�squeda y clasificaci�n bibliogr�fica o documental}, pudiendo citarse como ejemplo la publicaci�n en \gls{RDF} del \textit{Diccionario de Lugares Geogr�ficos} y el \textit{Diccionario y Tesauro de Materias} del Patrimonio Cultural de Espa�a, gestionado en la actualidad por el Ministerio de Educaci�n, Cultura y Deporte del Gobierno de Espa�a\footnote{\url{https://www.mecd.gob.es/cultura-mecd/areas-cultura/museos/destacados/2015/tesauros.html}}.

\section{\gls{SPARQL}} \label{sec:sparql}

Tal y como \gls{SQL} provee (al menos, relativamente) un de un est�ndar de consulta  para las bases de datos relacionales, \gls{SPARQL} ofrece un lenguaje de consulta estandarizado para grafos \gls{RDF}. 

\gls{SPARQL} es un acr�nimo recursivo para \textit{\gls{SPARQL} Protocol And \gls{RDF} Query Language}\cite{Seaborne:08:SQL}; es a la vez un lenguaje de consulta de grafos \gls{RDF} y un protocolo, dado que permite exponer servicios web que acepten consultas en dicho lenguaje.

\noindent\begin{minipage}{\linewidth}
Entre sus caracter�sticas, cabe citar:

\begin{itemize}  
	\item Recuperaci�n de valores de datos estructurados o semi-estructurados.
	\item Exploraci�n de conjuntos de datos a trav�s de consultas sobre relaciones desconocidas.
	\item Ejecuci�n de uniones complejas sobre bases de datos dispersas en una �nica consulta.
	\item Transformaci�n de datos \gls{RDF} desde un vocabulario a otro.
\end{itemize}
\end{minipage}

La estructura b�sica de una consulta \gls{SPARQL} es la siguiente:

\begin{lstlisting}[basicstyle=\tiny, caption={Estructura b�sica de consulta \gls{SPARQL}.},captionpos=b]

# declaraciones de prefijos
PREFIX foo: <http://example.com/resources/>
...
# definici�n del conjunto de datos
FROM ...
# cl�usula de resultado
SELECT ...
# patr�n de la consulta
WHERE {
	...
}
# modificadores de la consulta
ORDER BY ...
\end{lstlisting}

\pagebreak

No es objeto de este proyecto profundizar en la sintaxis del lenguaje, pero s� se incluye a continuaci�n un ejemplo de consulta simple:

\begin{lstlisting}[basicstyle=\tiny, caption={Consulta devolviendo todos los nombres de personas mencionadas en un grafo.},captionpos=b]
PREFIX foaf:  <http://xmlns.com/foaf/0.1/>
SELECT ?name
WHERE {
	?person foaf:name ?name .
}
\end{lstlisting} 

\cleardoublepage{}

\chapter{Motivaci�n y objetivos}

Conocer los motivos que impulsan a llevar a cabo una determinada acci�n facilita la empat�a entre seres humanos, promueve el contagio de ilusiones y ayuda a comprender. El presente cap�tulo cuenta con dos secciones bien diferenciadas, explicando la motivaci�n que llev� a la elecci�n de este proyecto y los objetivos que se intentaban conseguir con el mismo.

\section{Motivaci�n}

En el �mbito acad�mico, la Web Sem�ntica se contextualiza e integra en materias como la \textit{Gesti�n Avanzada de la Informaci�n y el Conocimiento}, donde las Tecnolog�as de la Informaci�n se insertan con la Inteligencia Artificial con el objeto de modelar conocimiento humano para un posterior procesamiento y aprendizaje autom�tico.

Las tecnolog�as de la Web Sem�ntica se caracterizan por un cambio de paradigma bastante brusco con respecto a otras tecnolog�as de manipulaci�n de datos con las que un alumno muy probablemente pueda estar familiarizado a la hora de introducirse en este nuevo mundo. Es cierto que pueden establecerse ciertas comparativas conceptuales para facilitar su comprensi�n; como ejemplo, el cuadro \ref{comprelsem} equipara la Web Sem�ntica con un sistema basado en bases de datos relacionales.

\vspace*{\baselineskip}
\begin{center}
	\begin{table}[htb]
		\centering
		\begin{tabular}{p{5cm}p{5cm}}
			\toprule
			Mundo Relacional & Mundo Sem�ntico \\ \midrule
		    Registro de tabla & Nodo \gls{RDF} \\ \midrule
		    Columna de tabla & \gls{RDF} propertyType \\ \midrule
			Celda de tabla & Valor \\ \midrule
			Consulta \gls{SQL} & Consulta \gls{SPARQL} \\ \midrule
			Modelo de datos & Ontolog�a \\ 
			\bottomrule
		\end{tabular}
		\caption{Comparativa paradigma relacional vs sem�ntico.}
		\label{comprelsem}
	\end{table}
\end{center}

\pagebreak

Sin embargo, determinados indicadores pueden dar lugar a entender que la interiorizaci�n de los conceptos asociados a la Web Sem�ntica no es sencilla. Por ejemplo: su uso casi exclusivamente acad�mico con poca penetraci�n en el mundo empresarial, su lenta velocidad de propagaci�n, su soporte limitado en determinadas plataformas tecnol�gicas o incluso el t�mido inter�s mostrado en el �mbito p�blico (excluyendo honrosas excepciones como el acad�mico o universitario, las iniciativas de datos abiertos o lo relativo a la biblioteconom�a). Un ejemplo muy concreto, real y tangible de esta escasa penetraci�n es el reducido n�mero de art�culos (cuatro) presentado en \textit{workshops} como \textit{SemStats}\footnote{\textit{SemStats} es un \textit{workshop} de celebraci�n anual centrado en el estado actual de las tecnolog�as de la Web Sem�ntica aplicadas a la estad�stica p�blica. Ver \url{http://semstats.org/}}\cite{semstats2018}, de cuyo \textit{Program Comittee} este autor forma parte:


\begin{figure}[h]
	\centering
	\includegraphics[scale=0.25]{semstats2018}
	\caption{Trabajos enviados a \textit{Semstats} 2018.}
\end{figure}

En este caso, resulta cuando menos sorprendente que las Oficinas Estad�sticas P�blicas, encargadas de generar y publicar datos estad�sticos sobre econom�a, poblaci�n y sociedad en general, no est�n mostrando mayor inter�s a la hora no ya de publicar sus datos en formatos enlazables, sino de construir aplicaciones sem�nticas para explotar y relacionar mejor estos datos.

Es esta \textbf{situaci�n de dificultad en el aprendizaje y en la interiorizaci�n de la utilidad de las tecnolog�as de la Web Sem�ntica la que motiva}, en �ltima instancia, la idea de desarrollar herramientas que acerquen este tipo de tecnolog�as al p�blico en general y faciliten una toma de contacto gradual con las mismas, ayudando as� a su divulgaci�n. 

\section{Objetivos} \label{objetivos}

Si bien es cierto que existen, por una parte, varias aproximaciones gratuitas en la red a modo de \textit{playground} que permiten el lanzamiento de consultas \gls{SPARQL} a determinados \textit{endpoints} y por otra, editores y herramientas de modelado del calibre de \textit{Prot�g�}\footnote{\label{protegefn}Un editor de ontolog�as y marco de trabajo para construir sistemas inteligentes. Ver \url{https://protege.stanford.edu/}} (analizados con detalle en la secci�n \ref{trabajosprevios}), no parece haber en el mercado referencias de aplicaciones que ofrezcan simult�neamente subconjuntos reducidos de ambas funcionalidades. Llenar este hueco y \textbf{conseguir una herramienta formativa �nica de modelado y consulta en tecnolog�as de la Web Sem�ntica }que permita a los equipos docentes de las asignaturas relacionadas con tecnolog�as de la Web Sem�ntica ofrecer a los alumnos (o a cualquier no iniciado en estas tecnolog�as) un \textit{playground} o entorno de pruebas que les ayude a consolidar su aprendizaje te�rico con una base pr�ctica adaptada al mundo real \textbf{es el objetivo principal de este proyecto}.

De forma m�s espec�fica, es posible enumerar los siguientes objetivos detallados:

\begin{enumerate}  
	\item  Facilitar el aprendizaje y la \textbf{familiarizaci�n con tecnolog�as de la Web Sem�ntica} a usuarios inexpertos.
	\item Reducir al m�ximo los requisitos de uso de la herramienta para alcanzar el mayor p�blico objetivo posible.
	\item Ofrecer a los equipos docentes la posibilidad de \textbf{distribuir actividades formativas} auto-contenidas para que los alumnos trabajen sobre ellas en la herramienta.
	\item Agrupar en un solo producto \textbf{funcionalidades b�sicas de edici�n y consulta sobre grafos \gls{RDF}}, de cara a cubrir las expectativas educativas en este �mbito.
	\item Permitir a los alumnos comprobar \textit{in situ} los efectos de sus acciones sobre un grafo \gls{RDF} y ofrecerles la \textbf{flexibilidad suficiente} como para permitirles dar rienda suelta a su creatividad e inquietudes.
	
	Y por �ltimo, y no menos importante, 
	
	\item \textbf{Obtener un \gls{MVP}}\footnote{Un producto con las suficientes caracter�sticas como para satisfacer a sus usuarios tempranos.} que pueda ser evolucionado y poner su c�digo a disposici�n de la comunidad educativa para revertir sobre ella su valor aportado.
\end{enumerate}
\cleardoublepage{}


\chapter{Trabajos previos, estado actual y recursos}


\section{Trabajos previos} \label{trabajosprevios}

Para enfocar el presente proyecto se llev� a cabo un estudio previo de los trabajos existentes que pudieran estar relacionado con los objetivos del mismo. En las siguientes subsecciones se analizan las familias de herramientas analizadas.

\subsection{Herramientas de consulta}

La herramienta de consulta web que se asemeja m�s a los resultados obtenidos con este proyecto es el SPARQL playground desarrollado por el \textit{Swiss Institute of Bioinformatics}\footnote{https://www.isb-sib.ch}.

Se trata de una aplicaci�n web\textit{ standalone} y multiplataforma cuyo prop�sito fundamental es el aprendizaje de SPARQL. Sus autores proporcionan una demo \textit{online}\footnote{http://sparql-playground.sib.swiss/} adem�s de documentaci�n y consultas de ejemplo para que cualquiera pueda familiarizarse con este lenguaje de consulta.

Si bien sus funcionalidades de consulta son muy completas, carece de facilidades de edici�n de grafos propios o de consulta a \textit{endpoints} externos, por ejemplo.

\begin{figure}[h]
	\centering
	\includegraphics[scale=0.25]{sparql_playground}
	\caption{Interfaz de SPARQL \textit{playground}}
\end{figure}

Existan tambi�n bibliotecas de consulta para el \textit{frontend}, como Jassa (\textit{JAvascript Suite for Sparql Access})\footnote{http://aksw.org/Projects/Jassa.html}. Se trata de una biblioteca modular que comprende desde una API RDF sobre una capa de abstracci�n de servicios hasta una capa de navegaci�n por facetas. Sin embargo, desarrolla un modelo propio del grafo RDF no basado en est�ndares y se integra con versiones muy antiguas de Angular, lo que hace que pierda inter�s para su uso en producci�n.

\subsection{Herramientas de modelado}

\subsubsection{Herramientas web}

Existen pocas herramientas con funcionalidades de edici�n de grafos RDF en la Web. Al margen de la versi�n web de Prot�g�, cuya versi�n de escritorio (m�s representativa) se comenta en el punto \ref{deescritorio}, el resto de trabajos encontrados son fundamentalmente bibliotecas o paquetes de edici�n de formularios, como rdfforms\footnote{https://rdforms.org/}. Se trata de una biblioteca Javascript cuyo prop�sito es facilitar la construcci�n de editores RDF basados en formularios en un entorno web. Para ello, se basa en un mecanismo de plantillas que describe c�mo generar un formulario HTML y c�mo mapear expresiones espec�ficas en un grafo RDF a sus correspondientes campos.

Unas pruebas iniciales con la biblioteca sugirieron falta de madurez por fallos en su funcionamiento en un navegador. Adem�s, la biblioteca implementa su propio modelo de grafo RDF, que no est� relacionado con los est�ndares actuales (ver secci�n \ref{estadoactual}).

\begin{figure}[h]
	\centering
	\includegraphics[scale=0.25]{rdfforms}
	\caption{Ejemplo de formulario desarrollado con rdfforms}
\end{figure}

\subsubsection{Herramientas de escritorio} \label{deescritorio}

La herramienta de modelado de ontolog�as en RDF por antonomasia es Prot�g�, un editor de ontolog�as \textit{opensource} gratuito y un marco de trabajo para construir sistemas inteligentes soportados por una fuerte comunidad de usuarios acad�micos, gubernamentales y corporativos. Se utiliza en �reas tan diversas como la biomedicina, el comercio electr�nico, la predicci�n metereol�gica o el modelado organizacional.

Prot�g� se instala como una aplicaci�n de escritorio multiplataforma y tiene una arquitectura basada en \textit{plugins} o complementos que permite extender f�cilmente su funcionalidad. Tiene soporte para razonadores sobre las distintas ontolog�as que se modelan y facilidades de representaci�n y edici�n de grafos RDF.

De cara a un primer encuentro con las tecnolog�as de la Web Sem�ntica, puede parecer abrumador por su gran flexibilidad y potencia. Adem�s, no cuenta con facilidades de consulta sobre los grafos modelados.

\begin{figure}[h]
	\centering
	\includegraphics[scale=0.25]{protege}
	\caption{Interfaz de Prot�g�}
\end{figure}


\section{Estado actual} \label{estadoactual}

El estado actual (o \textit{state of the art}, del ingl�s) de las tecnolog�as de la Web Sem�ntica alcanza distintos niveles de madurez en sus implementaciones. Complementando el listado de herramientas ya presentado en el estudio de trabajos previos (\ref{trabajosprevios}), se analizan a continuaci�n las plataformas o bibliotecas m�s representativas dentro de las tecnolog�as m�s utilizadas.

\subsection{Java}

As�, para la plataforma Java se cuenta con Apache Jena{\cite{jena01}}, un marco de trabajo o \textit{framework} \textit{opensource} para construir aplicaciones para la Web Sem�ntica o sobre datos enlazados. De entre sus componentes y funcionalidades, cabe destacar:

\begin{itemize} 
\item \textbf{TDB}: una base de datos de tripletas nativa y de alto rendimiento, con excelente integraci�n con el resto de APIs de Jena.
\item \textbf{ARQ}: un motor SPARQL compatible con su versi�n 1.1 que soporta consultas federadas y b�squeda por texto libre.
\item \textbf{API RDF}: una API principal que permite interactuar con RDF para crear y leer grafos y tripletas, as� como serializarlas a los formatos m�s comunes (Turtle, XML, etc.).
\item \textbf{Fuseki}: un \textit{endpoint} SPARQL que permite exponer el grafo de trabajo y ofrece interacci�n tipo REST con tripletas RDF.
\item \textbf{Otras APIs}: como las de ontolog�as e inferencias, que permiten a�adir m�s sem�ntica al modelo y razonar sobre reglas por defecto o personalizadas.
\end{itemize}  

Apache Jena es una soluci�n robusta y muy utilizada en entornos acad�micos y de producci�n empresarial con m�s de quince a�os de vida.

\subsection{Python}

RDFLib\footnote{https://github.com/RDFLib/rdflib} es una biblioteca \textit{opensource} ligera pero funcionalmente completa para trabajar con RDF desde plataformas Python. Permite a las aplicaciones acceder a estructuras RDF a trav�s de construcciones idiom�ticas en Python, lo que facilita un acercamiento de la tecnolog�a al programador experimentado; por ejemplo, un grafo no es m�s que una colecci�n de tripletas \textit{<sujeto, predicado, objeto>}. Entre el resto de sus caracter�sticas, destacan:

\begin{itemize}
	\item Contiene procesadores y serializadores para XML, N3, Turtle, RDFa, etc.
	\item Presenta una interfaz para un grafo que puede soportarse sobre multitud de implementaciones de almacenes.
	\item Incluye una implementaci�n de SPARQL v1.1.
	\item Presenta una arquitectura modular basada en \textit{plugins} o complementos.
\end{itemize} 

Su desarrollo es estable y ha sido utilizada en referencias bibliogr�ficas como \textit{Programming the Semantic Web}\cite{evanstaylor01}, de\textit{ O'Reilly}.

\subsection{Javascript}

Dado que uno de los objetivos del proyecto es reducir al m�ximo los requisitos de uso de la herramienta para alcanzar el mayor p�blico objetivo posible (ver \ref{objetivos}), Javascript se presenta como una tecnolog�a de sumo inter�s, ya que permite producir aplicaciones ejecutables al cien por cien en un navegador web est�ndar.

Sin embargo, el \textbf{estado actual de implementaci�n de RDF en Javascript puede calificarse de inmaduro y e inestable}. Se han llevado a cabo intentos aislados de construir APIs RDF y herramientas para gesti�n de grafos y consultas SPARQL, como las citadas en la secci�n \ref{trabajosprevios}, pero no existe un est�ndar �nico consolidado (a�n).

En el W3C puede consultarse la mejor comparativa existente de bibliotecas desarrolladas en Javascript para trabajar con RDF\cite{rdfjscomparison}.

Sin duda, el esfuerzo m�s destacable hasta ahora para conseguir un est�ndar de RDF en Javascript es el proyecto rdflib.js\footnote{https://github.com/linkeddata/rdflib.js}, liderado por el mism�simo Tim Berners-Lee. Se trata de una biblioteca que proporciona una API local para lanzar consultas a un almac�n RDF, soporta parte de la especificaci�n SPARQL y permite importar y escribir formatos como RDF/XML, Turtle y N3. 

En el propio repositorio de rdflib.js se cita al \textit{ RDFJS Representation Task Force}\footnote{https://github.com/rdfjs/representation-task-force}, un grupo dentro del \textit{RDFJS Community Group}\footnote{https://www.w3.org/community/rdfjs/} encargado de dise�ar las especificaciones de interfaz con el objetivo de conseguir que las distintas implementaciones de conceptos RDF en Javascript sean interoperables. La especificaci�n de la interfaz existe en forma de borrador a fecha de 14 de Agosto de 2017\footnote{http://rdf.js.org/}. En ella se definen interfaces para t�rminos, cuaternas, factor�as de datos, nodos en blanco, literales, etc. as� como para \textit{streams} de Javascript.

Consultando el sitio web del \textit{Community Group} se puede encontrar una noticia del 23 de Abril de 2018 (publicada, por tanto, en pleno desarrollo de este proyecto) anunciando la creaci�n de RDF.js, que tratar� de unir los esfuerzos de distintas entidades (como la Universidad de Gante o el MIT) y alinear y publicar bajo un �nico paraguas proyectos como:

\pagebreak

\begin{itemize}
	\item El citado rdflib.js
	\item RDF-Ext, una biblioteca Javascript para trabajar con RDF y datos enlazados que cumple con la especificaci�n del RDFJS Representation Task Force.
	\item N3.js, una biblioteca que implementa la especificaci�n de bajo nivel RDF.js y permite leer, procesar, almacenar y exportar ternas RDF de forma as�ncrona en Javascript.
\end{itemize} 

Ante esta situaci�n confusa y repleta de distintas implementaciones, se opt� por utilizar canales de comunicaci�n directos con sus autores. Existen varios grupos de discusi�n en Gitter, tales como:

\begin{itemize}
	\item rdfjs/public\footnote{https://gitter.im/rdfjs/public}
	\item rdf-ext/discussions\footnote{https://gitter.im/rdf-ext/discussions}
\end{itemize} 

En ellos, el autor del proyecto tuvo la oportunidad de recibir indicaciones directamente de Thomas Bergwinkl, 


\section{Recursos}
\cleardoublepage{}

\chapter{Metodolog�a y Planificaci�n}

La metodolog�a y la planificaci�n son dos labores obligadas en cualquier proyecto de ingenier�a. No obstante, cabe aqu� citar la siguiente frase atribuida a Dwight D. Eissenhower: \blockquote[{\cite{eissenhower01}}]{Plans are worthless, but planning is everything.} Hay que tener en cuenta que una planificaci�n es tan s�lo una gu�a inicial para conseguir una organizaci�n m�s depurada durante el tiempo de desarrollo. La planificaci�n debe ser una herramienta flexible de ayuda, y no un instrumento r�gido que condicione otras caracter�sticas del sistema tales como su calidad. En las siguientes secciones se analizan con m�s detalle estos asuntos.

\section{Metodolog�a}

Para acometer este proyecto, se han valorado dos familias de metodolog�as de desarrollo de software:

\begin{enumerate}
	\item Metodolog�as en cascada (\textit{Waterfall})
	\item Metodolog�as �giles (incrementales e iterativas)
\end{enumerate} 

La metodolog�a de desarrollo en cascada surgi� como idea en un art�culo de Winston W. Royce en 1970{\cite{royce01}}. Hist�ricamente, este modelo se ha extendido tanto en �mbitos acad�micos como profesionales, siendo estas sus principales caracter�sticas:

\begin{itemize}
	\item Gesti�n predictiva de proyectos llevada al software.
	\item Toma como modelo la forma de proceder en el resto de ingenier�as.
	\item Intenta llenar el vac�o del \textit{code \& fix}.
	\item Cada fase se realiza, en principio, una �nica vez.
	\item Cada fase produce un entregable que ser� entrada de la siguiente.
	\item Los entregables no son, en principio, modificables.
\end{itemize}

Es decir, se basa en la separaci�n entre dise�o y construcci�n (o entre creatividad y repetici�n).

La propuesta de Royce,tal y como se desprende de la lectura del art�culo original, describ�a el modelo en cascada como la \blockquote[{\cite{royce01}}]{descripci�n m�s simple} que solo funcionar�a para los proyectos m�s sencillos. Ir�nicamente, este mensaje malentendido ha sido el origen de la popularidad de la metodolog�a en cascada, que hoy en d�a se sigue promoviendo en muchos casos por inercia, desconocimiento, comodidad o ilusi�n de control sobre el proyecto.
 
Lamentablemente, en la Ingenier�a de Software los pesos de dise�o y construcci�n est�n invertidos con respecto a otras ingenier�as, siendo el software un dominio de cambio y alta inestabilidad. El desarrollo de software es, intr�nsecamente, una labor creativa; y la creatividad no es f�cilmente predecible. Esto ha dado lugar a que los desarrollos tradicionales adolezcan de ciertos problemas{\cite{larman01}}:

\begin{itemize}
	\item Existencia de muchos requisitos vagos o especulativos y dise�o detallado por adelantado.
	\item Est�n fuertemente asociados con las tasas de fallo m�s altas en proyectos.
	\item Se encuentran promovidos hist�ricamente por creencia m�s que por evidencia estad�stica significativa.
	\item Su rigidez incrementa el riesgo de fracaso, pospuesto hasta las fases finales del proyecto.
	\item Asume que las especificaciones son predecibles, estables y completas.
	\item Pospone integraci�n y pruebas hasta fases tard�as.
	\item Se basa en estimaciones y planificaci�n ``fiables''.
\end{itemize}

Entre los estudios que ratifican las afirmaciones anteriores, cabe citar, entre otros, los siguientes:

\begin{itemize}
	\item Informe Chaos 2015\cite{chaos15}.
	\item \textit{Dr. Dobb's Journal article The Non'Existent Software Crisis: Debunking the Chaos Report}\cite{ambler01}.
	\item Encuesta de Gartner\cite{mieritz01}.
\end{itemize}

Por tanto, atendiendo a esta exposici�n y considerando que el proyecto en cuesti�n presentaba un alto nivel de incertidumbre debido a su alto componente en investigaci�n del estado tecnol�gico actual, se ha optado por utilizar un enfoque metodol�gico incremental e iterativo, puesto que:

\begin{itemize}
	\item Facilita llevar a cabo proyectos peque�os.
	\item Fomenta la interacci�n entre el desarrollador y el usuario.
	\item Fuerza a que los inevitables cambios en requisitos sucedan en fases tempranas del proyecto.
\end{itemize}

Entre las caracter�sticas de este enfoque incremental e iterativo, es posible citar:

\begin{itemize}
	\item Se trabaja sobre subconjuntos de funcionalidad (\textit{features}).
	\item Los incrementos permiten a�adir funcionalidad al producto (mejora del proceso).
	\item Las iteraciones permiten redise�ar, revisar y refactorizar el producto (mejora del producto).
	\item Se basa en entregas frecuentes y ciclos prueba/error.
	\item Ofrece flexibilidad a la hora de gestionar el cambio.
\end{itemize}

La referencia m�s clara en este �mbito es \textit{Extreme Programming}, de Kent Beck\cite{beck01}. \textit{Extreme Programming} es \blockquote[{\cite{royce01}}]{un estilo de desarrollo de software centrado en la aplicaci�n excelente de t�cnicas de programaci�n, comunicaci�n clara y trabajo en equipo que permite conseguir objetivos antes impensables}. Se trata de una metodolog�a basada en valores como la comunicaci�n, realimentaci�n, simplicidad, valent�a y respeto, soportada sobre un cuerpo de pr�cticas �tiles y con un conjunto de principios complementarios, adem�s de contar con una comunidad de usuarios que comparte todo lo anterior.

Su aplicaci�n al desarrollo de este proyecto no ha sido estricta; por ejemplo, no se han definido ciclos estrictos por la propia naturaleza inestable de la dedicaci�n al desarrollo del mismo, pero s� que ha realizado un dise�o evolutivo soportado sobre un n�mero suficiente de pruebas unitarias as� como una planificaci�n incremental y adaptativa a las problem�ticas que iban surgiendo.

Prueba del acierto a la hora de elegir esta metodolog�a es el cambio de necesidades y requisitos funcionales que tuvo lugar en la reuni�n presencial mantenida con el tutor, donde \textbf{la metodolog�a aport� que no fuera necesario descartar ning�n desarrollo o trabajo realizado hasta la fecha}. Permiti� realizar una gesti�n del cambio efectiva, re-orientando el trabajo a tiempo sin impactar en el dise�o ni en la implementaci�n existente. Finalmente, tanto la organizaci�n como la planificaci�n de las tareas han sido lo suficientemente flexibles para conseguir un ritmo adecuado de desarrollo, reduciendo los puntos de bloqueo.

\section{Planificaci�n}


\subsection{Planificaci�n global}

Para realizar la planificaci�n del proyecto, y dada la metodolog�a de desarrollo incremental e iterativa elegida, no se ha seguido un modelo t�pico en fases de \textit{An�lisis, Dise�o, Implementaci�n, etc.} sino que se ha optado por un enfoque orientado a tareas relacionadas con la funcionalidad del producto final esperado.

\begin{figure}[h]
	\begin{ganttchart}[
		canvas/.append style={fill=none, draw=black!5, line width=.75pt},
		hgrid style/.style={draw=black!5, line width=.75pt},
		vgrid={*1{draw=black!5, line width=.75pt}},
		title/.style={draw=none, fill=none},
		title label font=\bfseries\footnotesize,
		title label node/.append style={below=4pt},
		include title in canvas=false,
		bar label font=\mdseries\small\color{black!70},
		bar label node/.append style={left=2cm},
		bar/.append style={draw=none, fill=black!63},
		bar incomplete/.append style={fill=barblue},
		bar progress label font=\mdseries\footnotesize\color{black!70},
		group/.append style={draw=none, fill=groupblue},
		group left shift=0,
		group right shift=0,
		group height=.5,
		group peaks tip position=0,
		group label node/.append style={left=.6cm},
		group progress label font=\bfseries\small,
		link/.style={-latex, line width=1.5pt, linkred},
		link label font=\scriptsize\bfseries,
		link label node/.append style={below left=-2pt and 0pt}
		]{1}{20}
		\gantttitle{Planificaci�n inicial}{20} \\
		\gantttitle{Febrero}{4}
		\gantttitle{Marzo}{4}
		\gantttitle{Abril}{4}
		\gantttitle{Mayo}{4}
		\gantttitle{Junio}{4}\\
		\gantttitle[
		title label node/.append style={below left=4pt and -3pt}
		]{Semana:\quad1}{1}
		\gantttitlelist{2,...,20}{1} \\
		\ganttgroup[]{Documentaci�n y arranque}{1}{4} \\
		\ganttgroup[]{Prueba de concepto}{4}{8} \\
		\ganttgroup[]{Desarrollo del proyecto}{8}{14} \\
		\ganttgroup[]{Resultados y memoria}{14}{20} \\
	\end{ganttchart}
	\caption{Planificaci�n inicial}\label{fig:gantt01}
\end{figure}

\pagebreak

La figura anterior (\ref{fig:gantt01}) muestra la planificaci�n propuesta para la elaboraci�n del anteproyecto.

En esta propuesta era clave la revisi�n de la prueba de concepto con el tutor para comprobar que efectivamente se hab�an comprendido las necesidades desprendidas del an�lisis y para poder continuar con la especificaci�n funcional de forma m�s refinada y precisa.

En la pr�ctica, la evoluci�n del proyecto ha sido bien distinta. En la figura \ref{fig:gantt02} se puede comprobar cu�l ha sido la planificaci�n efectiva, producto esta de cambios sobre la inicial para resolver los distintos imprevistos que han ido surgiendo durante la ejecuci�n del mismo.

\begin{figure}
	\begin{changemargin}{-1cm}{-1cm}
		\begin{tikzpicture}[y=0.50cm] 
		\begin{ganttchart}[
		y unit chart=0.8cm,
		canvas/.append style={fill=none, draw=black!5, line width=.75pt},
		hgrid style/.style={draw=black!5, line width=.75pt},
		vgrid={*1{draw=black!5, line width=.75pt}},
		title/.style={draw=none, fill=none},
		title label font=\bfseries\footnotesize,
		title label node/.append style={below=4pt},
		include title in canvas=false,
		bar label font=\mdseries\tiny\color{black!70},
		bar label node/.append style={left=0.5cm},
		bar/.append style={draw=none, fill=barblue!63},
		bar incomplete/.append style={fill=barblue},
		bar progress label font=\mdseries\footnotesize\color{black!70},
		group incomplete/.append style={fill=groupblue},
		group/.append style={draw=none, fill=groupblue},
		group left shift=0,
		group right shift=0,
		group height=.5,
		group peaks tip position=0,
		group label node/.append style={left=.6cm},
		group label font=\bfseries\tiny,
		milestone label font=\bfseries\tiny,
		link/.style={-latex, line width=1.5pt, linkred},
		link label font=\scriptsize\bfseries,
		link label node/.append style={below left=-2pt and -5pt}
		]{1}{28}
		\gantttitle{Planificaci�n efectiva}{28} \\
		\gantttitle{Febrero}{2}
		\gantttitle{Marzo}{4}
		\gantttitle{Abril}{4}
		\gantttitle{Mayo}{4}
		\gantttitle{Junio}{4}
		\gantttitle{Julio}{4}
		\gantttitle{Agosto}{4}
		\gantttitle{Septiembre}{2}\\
		\gantttitle[
		title label node/.append style={below left=4pt and -6pt}
		]{Semana:\quad1}{1}
		\gantttitlelist{2,...,28}{1} \\
		\ganttgroup[]{Formaci�n y arranque (FA)}{1}{6} \\
		\ganttbar[
		name=FA11
		]{\textbf{FA 1.1} RDF y OWL}{1}{2} \\
		\ganttbar[
		name=FA12
		]{\textbf{FA 1.2} JS ES6+, Vue}{3}{6} \\
		\ganttbar[
		name=FA13
		]{\textbf{FA 1.3} POC edici�n tripleta}{1}{6} \\
		\ganttmilestone{Prueba de concepto inicial}{6}{6} \\
		\ganttbar[
		name=FA14
		]{\textbf{FA 1.4} An�lisis de necesidades}{1}{6} \\
		\ganttgroup[]{Desarrollo de m�dulo de edici�n (ME)}{7}{20} \\
		\ganttbar[]{\textbf{ME 2.1} Plantilla Vue + testing}{7}{8} \\
		\ganttbar[]{\textbf{ME 2.2} Dise�o de componentes}{8}{9} \\
		\ganttbar[]{\textbf{ME 2.3} Implementaci�n CRUD clases}{9}{13} \\
		\ganttbar[]{\textbf{ME 2.4} Implementaci�n export/import}{12}{14} \\
		\ganttbar[]{\textbf{ME 2.5} Mejoras en visualizaci�n y uso }{14}{16} \\
		\ganttbar[]{\textbf{ME 2.6} Refactorizaci�n y generalizaci�n }{16}{20} \\
		\ganttbar[]{\textbf{ME 2.7} Pruebas}{7}{20} \\
		\ganttmilestone{Reuni�n presencial}{20}{20} \\
		\ganttgroup[]{Desarrollo de m�dulos playground}{20}{25} \\
		\ganttbar[]{\textbf{ME 3.1} M�dulo de consulta SPARQL}{20}{23} \\
		\ganttbar[]{\textbf{ME 3.2} M�dulo de carga de actividades}{23}{24} \\
		\ganttbar[]{\textbf{ME 3.3} M�dulo de carga de vocabularios}{23}{24} \\
		\ganttbar[]{\textbf{ME 3.4} Mejoras en la importaci�n}{23}{24} \\
		\ganttbar[]{\textbf{ME 3.5} Revisi�n, refactoring, cleaning}{24}{25} \\
		\ganttbar[]{\textbf{ME 3.6} Pruebas}{20}{25} \\
		\ganttgroup[]{Redacci�n de memoria y cierre}{24}{28} \\
		\ganttbar[]{\textbf{ME 4.1} Formaci�n latex}{24}{24} \\
		\ganttbar[]{\textbf{ME 4.2} Redacci�n memoria}{24}{28} \\
		
		\end{ganttchart}
		\end{tikzpicture}
	\end{changemargin}
	\caption{Planificaci�n efectiva}\label{fig:gantt02}
\end{figure}

Con respecto a la planificaci�n inicial, se tiene que:

\begin{itemize}
	\item El proyecto se ha retrasado un total de ocho semanas.
	\item El per�odo de formaci�n llev� al menos dos semanas m�s de lo esperado (en realidad, el aprendizaje del marco de trabajo Vue ha estado presente a lo largo de pr�cticamente todo el desarrollo del proyecto).
	\item El desarrollo del producto ha consumido seis semanas m�s de las previstas inicialmente.
	\item La memoria se ha redactado en dos semanas menos con respecto a la estimaci�n inicial.
\end{itemize} 

Las desviaciones surgidas con respecto a la planificaci�n inicial tienen su explicaci�n en los siguientes motivos:

\begin{itemize}
	\item El alto grado de incertidumbre a la hora de estimar la planificaci�n inicial, dado que se desconoc�a cu�l era la situaci�n actual del ecosistema tecnol�gico de \textit{frontend} web y su integraci�n con las tecnolog�as de la Web Sem�ntica.
	\item La curva de aprendizaje de Vue, si bien es considerada m�s suave que la de sus competidores (React, Angular) fue mayor de lo esperado. El bajo grado de familiarizaci�n del autor del proyecto con las tecnolog�as de \textit{frontend} y, especialmente, \textit{Javascript ES6+}, no facilit� el aprendizaje.
	\item El\textbf{ bajo nivel de madurez de las bibliotecas existentes} para la manipulaci�n de RDF en \textit{Javascript}, as� como la heterogeneidad y poca estabilidad de estas implementaciones, ha suscitado muchas dudas sobre cu�les utilizar y c�mo enfocar su integraci�n con marcos de trabajo m�s maduros como Vue.
	\item La \textbf{pr�ctica ausencia de bibliotecas} o componentes de interacci�n con SPARQL \textbf{conformes a la especificaci�n est�ndar de la interfaz} \textit{rdf.js}{\cite{rdfjs01}} (que por otra parte, es un borrador de 2017) ha puesto en peligro la viabilidad del proyecto. La plataforma finalmente utilizada, Comunica{\cite{comunica01}}, a�n no tiene una versi�n estable muchos de sus m�dulos (concretamente, el \textit{endpoint} SPARQL utilizado para el proyecto no la tiene), con lo que ha sido necesario estar en contacto directo con sus autores y colaborar con ellos en la revisi�n de defectos o \textit{bugs} y dependiendo por tanto de sus tiempos de respuesta (hay que tener en cuenta que la plataforma es \textit{opensource} y por tanto no existe acuerdo de nivel de servicio alguno.)
	\item La dificultad existente en llevar a cabo un an�lisis de requisitos a trav�s de una plataforma online de colaboraci�n como puede ser Skype: para constatar este hecho no hay m�s que verificar el cambio de rumbo del proyecto una vez mantenida la reuni�n presencial, que sirvi� para definir objetivos m�s claros y comprender las necesidades del Departamento.
\end{itemize} 

A pesar de todo ello, el autor de este proyecto est� satisfecho con el nivel de conocimiento adquirido en el �mbito de todas las tecnolog�as empleadas y el tiempo consumido para obtener como resultado un producto desplegado en producci�n y listo para utilizar.


\subsection{Planificaci�n �gil} \label{subsec:agileplanning}

Si bien para la planificaci�n global del proyecto se han utilizado herramientas tales como el \textit{diagrama de Gantt}, para gestionar el trabajo del d�a a d�a otro enfoque ha sido necesario. En el contexto de metodolog�as �giles de desarrollo de software de tipo incremental e iterativo como \textit{Extreme Programming}, \textbf{las tareas de planificaci�n adquieren un car�cter adaptativo} muy distinto del que presenta una planificaci�n tradicional con una metodolog�a de desarrollo en cascada, por ejemplo.

La planificaci�n �gil es un proceso en continua evoluci�n, tambi�n iterativo e incremental como las metodolog�as a las que pertenece, basada en ciclos del tipo:

\begin{enumerate}
	\item A�adir tareas
	\item Estimar tareas
	\item Priorizar tareas
\end{enumerate}

En este caso, la estimaci�n de tareas se llev� a cabo de forma heur�stica, utilizando la experiencia del autor y el conocimiento del contexto existente en el momento de la estimaci�n. En base a ello, la priorizaci�n (ordenaci�n) de las tareas ten�a lugar inmediatamente, relegando a la estimaci�n a un segundo plano.

Para gestionar estas tareas, se ha optado por utilizar como herramienta Kanban\footnote{\url{https://es.atlassian.com/agile/kanban}}. Un tablero Kanban puede definirse como un dispositivo de se�alizaci�n que introduce el flujo de trabajo de un proceso a un ritmo manejable. Presenta las siguientes caracter�sticas:


\begin{itemize}
	\item Solo env�a trabajo cuando lo ordena el cliente o usuario (en este caso, el propio autor del proyecto)
	\item Indica espec�ficamente qu� trabajo debe hacerse.
	\item Controla la cantidad de trabajo en progreso.
	\item Regula las interrupciones y orquesta el ritmo de trabajo.
\end{itemize} 


B�sicamente, consiste en utilizar una tabla con varias columnas para visualizar el estado de una tarea a lo largo de las distintas fases que se consideren. Para el caso de la realizaci�n de este proyecto, se crearon las siguientes columnas:

\begin{center}
	\begin{table}[htb]
		\centering
		\begin{tabular}{p{3cm}p{10cm}}
			\toprule
			Columna & Descripci�n \\
			\toprule
			To-Do & Tareas por realizar en orden de prioridad descendente \\ \midrule
			Work in progress & Tareas realiz�ndose en un momento dado. No m�s de dos o tres. \\ \midrule
			Stand-by & Tareas a la espera por motivos ajenos al autor del proyecto. \\ \midrule
			Done & Tareas finalizadas. \\ \midrule
			Discarded & Tareas descartadas debido a cambios de dise�o, requisitos, etc. \\ \midrule
			Doubts & Dudas planteadas a lo largo del desarrollo del proyecto. \\ \midrule
			Resources & Recursos documentales en la web �tiles para del desarrollo del proyecto. \\
			\bottomrule
		\end{tabular}
		\caption{Dise�o del tablero Kanban}
		\label{tabkanban}
	\end{table}
\end{center}

\begin{figure}[h]
	\centering
	\includegraphics[scale=0.25]{trello}
	\caption{Ejemplo de tablero Trello}
\end{figure}


\cleardoublepage{}

\chapter{An�lisis}

La captura de requisitos es quiz� una de las labores m�s complicadas a la hora de desarrollar un sistema de informaci�n, debido a las ``diferencias de impedancia'' existentes entre un analista y un usuario de negocio. En el caso particular de este proyecto, el hecho de que el usuario fuera a su vez el Director del mismo ha supuesto una ayuda considerable. No obstante, a continuaci�n se detallan las estrategias que supusieron una mejora considerable en los niveles de concreci�n del alcance a delimitar.

\section{Captura y documentaci�n de requisitos}

\subsection{Captura de requisitos}

La t�cnica de captura de requisitos utilizada para este proyecto ha sido, fundamentalmente, \textbf{la entrevista} con el tutor. Se eligi� esta t�cnica por los siguientes motivos:

\begin{itemize}  
	\item Las entrevistas, bien a distancia o presenciales (y especialmente estas �ltimas), permiten una mayor implicaci�n del usuario en la captura de requisitos.
	\item Combinada con una maqueta o prueba de concepto, una entrevista presencial puede dar lugar a la aparici�n de nuevos requisitos de producto, cambios en las especificaciones e incluso en el enfoque y objetivos del mismo.
	\item Permite la pr�ctica de la escucha activa y la sugerencia de ideas por parte del analista, aportando un valor a�adido que enriquece la simple captura de requisitos.
	\item Es claramente la t�cnica m�s obvia, directa y accesible en el contexto de la realizaci�n del proyecto.
\end{itemize}

Concretamente, se han llevado a cabo varias entrevistas utilizando la plataforma colaborativa Skype y una presencial, en la que el autor de este proyecto se ha desplazado a la sede del departamento en Madrid con objeto de conseguir una comunicaci�n m�s fluida y un mayor entendimiento a la hora de consensuar las necesidades y funcionalidades requeridas del producto.

La primera entrevista a distancia propici� un intercambio de documentos e ideas que desemboc� en la elaboraci�n del documento del anteproyecto. El resto de entrevistas a trav�s de Skype sirvieron para concretar en mayor medida las tareas a realizar y el enfoque del proyecto. Sin embargo, no fue hasta que no tuvo lugar la reuni�n presencial cuando realmente se le dio al proyecto su orientaci�n final, con unos objetivos claramente definidos y una posibilidad para cumplir los hitos propuestos.

A continuaci�n se resumen todas las sesiones de captura de requisitos:

\vspace*{\baselineskip}
\begin{center}
\begin{table}[htb]
\centering
\begin{tabular}{p{1cm}p{2cm}p{10cm}}
	\toprule
	id & Fecha & Resumen de la entrevista \\
	\toprule
	1 & 18/10/17 & Primer contacto y comunicaci�n de ideas iniciales para la confecci�n del anteproyecto  \\ \midrule
	2 & 21/02/17 & Consolidaci�n de ideas y aportaci�n de m�s documentaci�n (v�deos sobre prototipos de cuaderno, documentos de texto con descripciones, etc.)  \\
	3 & 28/03/18 & Primera demo a modo de POC con un entorno capaz de a�adir tripletas.  \\ \midrule
	4 & 10/07/18 & Reuni�n presencial con demostraci�n \textit{in-situ} de los m�dulos de modelado e importaci�n/exportaci�n. Tiene lugar una tormenta de ideas y se enfoca el proyecto de otro modo, modificando sus objetivos hacia una herramienta formativa..  \\ \midrule
	5 & 4/08/18 & Revisi�n de los �ltimos avances con la integraci�n de un endpoint SPARQL en el frontend y planificaci�n del resto de funcionalidade requeridas.  \\ 
	\bottomrule
\end{tabular}
\caption{Sesiones de entrevistas}
\label{tabmeetings}
\end{table}
\end{center}

\subsection{Documentaci�n}

Para documentar la captura de requisitos, se utilizar� la t�cnica de casos de uso. Se descarta la incorporaci�n de diagramas UML de casos de uso, dado que dichos diagramas carecen de informaci�n esencial sobre los mismos (como qu� actor lleva a cabo cada paso, o notas sobre el orden de ejecuci�n de los pasos). Si bien pueden ser �tiles como resumen o �ndice de contenidos, se decide prescindir de ellos dado que el n�mero de casos de uso contemplados en el proyecto es manejable.

Se utilizar� una plantilla propuesta por Cockburn{\cite{cockburn01}}: el estilo RUP (\textit{Rational Unified Process}){\cite{rational01}}, atractivo y f�cil de seguir pese al elevado n�mero de apartados, modificado para plasmar los aspectos m�s relevantes del proyecto (por ejemplo, no se incluir� un campo \textit{�mbito} porque siempre va a estar referido al mismo sistema o aplicaci�n). El motivo de no utilizar una tabla es meramente subjetivo, ya que el autor de esta memoria opina que puede oscurecer el contenido.

La plantilla sigue la siguiente estructura:

\begin{enumerate}  
	\item Nombre del caso de uso
	\begin{enumerate}
		\item Descripci�n breve
		\item Actores, entre los que estar� el actor principal. Presentan comportamiento.
		\item Disparadores: acciones sobre el sistema que inician los casos de uso.
	\end{enumerate}
	\item Flujo de eventos
		\begin{enumerate}
		\item Flujo b�sico: escenario principal de �xito.
		\item Flujos alternativos: qu� puede pasar que no sea el flujo principal.
			\begin{enumerate}
			\item Condicion 1
			\item Condici�n 2
			\item \ldots
		\end{enumerate}
	\end{enumerate}
	\item Requisitos especiales (si se dieran): plataforma, etc.
	\item Precondiciones: qu� debe ser cierto antes de ejecutar el caso de uso.
	\item Postcondiciones: qu� debe ser cierto despu�s de ejecutar el caso de uso.
\end{enumerate}

Los requisitos fueron capturados inicialmente como notas manuscritas y convertidos en necesidades de alto nivel en la plataforma Trello. A partir de ah�, dichas necesidades se refinaron para dar lugar a la bater�a de casos de uso incluida en \ref{sec:53}.

\section{Necesidades}

La reuni�n presencial marc� un punto de inflexi�n en cuanto a objetivos del proyecto, lo que se traduce en un cambio de necesidades. Para reflejar la evoluci�n completa, se dividir� su captura en dos fases detalladas a continuaci�n:

\subsection{Captura inicial}

A continuaci�n se resumen, en lenguaje natural, las necesidades identificadas durante la primera fase de desarrollo del proyecto:
\begin{enumerate}
\item Permitir a usuarios de la UNED  de distintos colectivos etiquetar y generar sus propios cuadernos con informaci�n y metainformaci�n sem�ntica.
\item Permitir a dichos usuarios realizar consultas sobre sus cuadernos.
\item Desarrollar una interfaz web que permita al usuario gestionar tripletas RDF.
\item Desarrollar un m�dulo de generaci�n de consultas SPARQL a partir de consultas de lectura y escritura en formato JSON.
\item Desarrollar los correspondientes productos de inter�s para el usuario: consultas exportadas en forma diversa (CSV, JSON) o embebidas en una plantilla HTML significativa para el usuario.
\item Ofrecer la posibilidad de variar interfaces de entrada y exportadores en funci�n de los distintos colectivos de usuario utilizando los metadatos sobre cuadernos RDF previamente almacenados en una base de datos relacional.
\item Permitir mostrar en pantalla una serie de t�rminos como punto de partida que el usuario pueda utilizar para construir ternas RDF y relaciones entre ellas.
\item Permitir mostrar en pantalla una serie de t�rminos como punto de partida que el usuario pueda utilizar para construir ternas RDF y relaciones entre ellas.
\item Ofrecer al usuario una visualizaci�n sencilla y correcta de su modelo que proporciona una perspectiva adecuada sobre la que trabajar.
\item Presentar una interfaz de mantenimiento del grafo: vocabulario e instancias (conceptualizaci�n y poblamiento de una ontolog�a).
\item Importar y exportar informaci�n estructurada en formatos sem�nticos est�ndar.
\item Extender con vocabularios tales como SKOS y OWL. 
\end{enumerate}

\subsection{Captura final}

Una vez celebrada la reuni�n presencial, se decidi� darle otro enfoque al proyecto. Si bien la idea inicial era desarrollar un sistema que permitiese generar cuadernos a trav�s de la manipulaci�n de grafos, una vez presentada una maqueta o prueba de concepto con funcionalidades b�sicas de modelado el tutor propuso convertir la herramienta en un \textit{playground} o sistema de realizaci�n de actividades acad�micas con un enfoque docente orientado a facilitar el aprendizaje de las tecnolog�as de la Web Sem�ntica (b�sicamente RDF y SPARQL) a personas con poco o ning�n contacto con estas materias (por ejemplo, alumnos de los Grados de Ingenier�a Inform�tica o en Tecnolog�as de la Informaci�n de la UNED).

Este nuevo enfoque se tradujo en la siguiente instant�nea de necesidades de alto nivel:

\begin{enumerate}
\item Permitir el modelado sem�ntico con edici�n CRUD de clases, subclases y propiedades de clases (anotaciones b�sicas).
\item Permitir la edici�n CRUD de relaciones o propiedades, subpropiedades, etc.
\item Ofrecer mecanismos de poblamiento del grafo.
\item Permitir el lanzamiento de consultas SPARQL sobre el grafo local y visualizaci�n de resultados.
\item Permitir el lanzamiento de consultas SPARQL sobre endpoints remotos y visualizaci�n de resultados.
\item Ofrecer funcionalidad de carga de consultas SPARQL predefinidas desde archivo de texto.
\item Incorporar m�dulo para a�adir un texto con la definici�n de la actividad a realizar en formato Markdown.
\item Permitir la importaci�n de tripletas desde archivos o URL.
\item Permitir la incorporaci�n (append) de tripletas al grafo local desde archivos o URL.
\item Ofrecer un panel para cargar en el grafo vocabularios comunes predefinidos.
\item Presentar una interfaz f�cil de usar y responsiva para el usuario, con una gesti�n de errores adecuada y suficiente.
\end{enumerate}

\section{Casos de uso} \label{sec:53}

\subsection{Dar de alta una nueva clase}\label{sec:dar-de-alta-una-nueva-clase}

\subsubsection{Descripci�n breve}

Este caso de uso permite a un usuario a�adir una nueva clase a su grafo a trav�s de una tripleta que tendr� como sujeto el nombre de la clase, como objeto el recurso owl:Class y como predicado el recurso rdf:type. 

\subsubsection{Actores}

El actor principal es el usuario de la aplicaci�n (fundamentalmente dirigida a un alumno, pero podr�a ser un profesor o cualquier otro usuario).

\subsubsection{Disparadores}

El caso de usa comienza cuando el usuario pulsa en el bot�n flotante + asociado a la tarjeta del listado de clases.

\subsubsection{Flujo de eventos}
\begin{enumerate}
	\item Flujo b�sico
	\begin{enumerate}
		\item El usuario pulsa en el bot�n flotante + asociado a la tarjeta del listado de clases.
		\item Aparece un cuadro de di�logo donde el usuario debe introducir el nombre de la clase.
		\item El usuario introduce el URI del recurso que quiere dar de alta como clase y pulsa en guardar.
		\item El nuevo recurso de tipo clase se crea y aparece en el listado de clases.
	\end{enumerate}
	\item Flujo alternativo 1
\begin{enumerate}
	\item El usuario pulsa en el bot�n flotante + asociado a la tarjeta del listado de clases.
	\item Aparece un cuadro de di�logo donde el usuario debe introducir el nombre de la clase.
	\item El usuario introduce (o no) el URI del recurso que quiere dar de alta como clase y pulsa fuera del cuadro de di�logo.
	\item El cuadro de di�logo se cierra y no hay ning�n cambio en los recursos.
\end{enumerate}
	\item Flujo alternativo 2
	\begin{enumerate}
		\item El usuario pulsa en el bot�n flotante + asociado a la tarjeta del listado de clases.
		\item Aparece un cuadro de di�logo donde el usuario debe introducir el nombre de la clase.
		\item El usuario introduce el URI del recurso que quiere dar de alta como clase y pulsa en guardar.
		\item Ocurre un error en el guardado del recurso que le es comunicado al usuario a trav�s de una notificaci�n en pantalla.
	\end{enumerate}

\end{enumerate}

\subsubsection{Requisitos especiales}

Ninguno.

\subsubsection{Precondiciones}

El usuario debe haber navegado previamente hasta la secci�n de modelado, bien a trav�s de la pantalla de bienvenida, bien a trav�s de la barra de navegaci�n lateral.

\subsubsection{Postcondiciones}

Ninguna.

\subsubsection{Extensiones}

Ninguna.

\subsection{Editar una clase existente}\label{sec:editar-una-clase-existente}

\subsubsection{Descripci�n breve}

Este caso de uso permite a un usuario editar el recurso asociado a una clase existente en su grafo y cambiar su URI.

\subsubsection{Actores}

El actor principal es el usuario de la aplicaci�n (fundamentalmente dirigida a un alumno, pero podr�a ser un profesor o cualquier otro usuario).

\subsubsection{Disparadores}

El caso de usa comienza cuando el usuario pulsa en el icono de men� asociado a la clase que quiere editar y selecciona la acci�n ``Editar''.

\subsubsection{Flujo de eventos}
\begin{enumerate}
	\item Flujo b�sico
	\begin{enumerate}
		\item El usuario pulsa en el icono de men� asociado a la clase que quiere editar y selecciona la acci�n ``Editar''.
		\item Aparece un cuadro de di�logo con el URI del recurso actual que el usuario puede modificar.
		\item El usuario modifica (o no) el URI del recurso  y pulsa en guardar.
		\item El recurso ha sido modificado y su cambio se ve reflejado en el listado.
	\end{enumerate}
\item Flujo alternativo 1
\begin{enumerate}
	\item El usuario pulsa en el icono de men� asociado a la clase que quiere editar y selecciona la acci�n ``Editar''.
	\item Aparece un cuadro de di�logo con el URI del recurso actual que el usuario puede modificar.
	\item El usuario modifica (o no) el URI del recurso  y pulsa fuera del cuadro de di�logo.
	\item El cuadro de di�logo se cierra y no hay ning�n cambio en los recursos.
\end{enumerate}
	\item Flujo alternativo 2
	\begin{enumerate}
		\item El usuario pulsa en el icono de men� asociado a la clase que quiere editar y selecciona la acci�n ``Editar''.
		\item Aparece un cuadro de di�logo con el URI del recurso actual que el usuario puede modificar.
		\item El usuario modifica (o no) el URI del recurso  y pulsa en guardar.
		\item Ocurre un error en el guardado del recurso que le es comunicado al usuario a trav�s de una notificaci�n en pantalla.
	\end{enumerate}
	
\end{enumerate}

\subsubsection{Requisitos especiales}

Ninguno.

\subsubsection{Precondiciones}

El usuario debe haber navegado previamente hasta la secci�n de modelado, bien a trav�s de la pantalla de bienvenida, bien a trav�s de la barra de navegaci�n lateral.

\subsubsection{Postcondiciones}

Ninguna.

\subsubsection{Extensiones}

Ninguna.

\subsection{Eliminar una clase existente}\label{sec:eliminar-una-clase-existente}

\subsubsection{Descripci�n breve}

Este caso de uso permite a un usuario eliminar el recurso asociado a una clase existente en su grafo.

\subsubsection{Actores}

El actor principal es el usuario de la aplicaci�n (fundamentalmente dirigida a un alumno, pero podr�a ser un profesor o cualquier otro usuario).

\subsubsection{Disparadores}

El caso de usa comienza cuando el usuario pulsa en el icono de men� asociado a la clase que quiere editar y selecciona la acci�n ``Eliminar''.

\subsubsection{Flujo de eventos}
\begin{enumerate}
	\item Flujo b�sico
	\begin{enumerate}
		\item El usuario pulsa en el icono de men� asociado a la clase que quiere editar y selecciona la acci�n ``Eliminar''.
		\item Aparece un cuadro de di�logo advirtiendo al usuario de que si confirma el cambio, no podr� volver a recuperar el recurso del grafo.
		\item El usuario elige confirmar.
		\item El recurso y su detalle asociado desaparecen del listado.
	\end{enumerate}
\item Flujo alternativo 1
\begin{enumerate}
	\item El usuario pulsa en el icono de men� asociado a la clase que quiere editar y selecciona la acci�n ``Eliminar''.
	\item Aparece un cuadro de di�logo advirtiendo al usuario de que si confirma el cambio, no podr� volver a recuperar el recurso del grafo.
	\item El usuario pulsa fuera del cuadro de di�logo.
	\item El cuadro de di�logo se cierra y no hay ning�n cambio en los recursos.
\end{enumerate}
	\item Flujo alternativo 2
	\begin{enumerate}
		\item El usuario pulsa en el icono de men� asociado a la clase que quiere editar y selecciona la acci�n ``Eliminar''.
	\item Aparece un cuadro de di�logo advirtiendo al usuario de que si confirma el cambio, no podr� volver a recuperar el recurso del grafo.
	\item El usuario elige cancelar.
	\item El recurso y su detalle siguen apareciendo y no son eliminados.
	\end{enumerate}
	\item Flujo alternativo 3
\begin{enumerate}
	\item El usuario pulsa en el icono de men� asociado a la clase que quiere editar y selecciona la acci�n ``Eliminar''.
	\item Aparece un cuadro de di�logo advirtiendo al usuario de que si confirma el cambio, no podr� volver a recuperar el recurso del grafo.
	\item El usuario elige confirmar.
	\item  Ocurre un error en el borrado del recurso que le es comunicado al usuario a trav�s de una notificaci�n en pantalla.
\end{enumerate}
	
\end{enumerate}

\subsubsection{Requisitos especiales}

Ninguno.

\subsubsection{Precondiciones}

El usuario debe haber navegado previamente hasta la secci�n de modelado, bien a trav�s de la pantalla de bienvenida, bien a trav�s de la barra de navegaci�n lateral.

\subsubsection{Postcondiciones}

Una vez eliminada un recurso de tipo clase, se eliminan tambi�n todas las tripletas que lo tengan como objeto o como sujeto (es decir, cualquier relaci�n con dicha clase).

\subsubsection{Extensiones}

Ninguna.

\subsection{Dar de alta una nueva anotaci�n o propiedad est�ndar de una clase}\label{sec:dar-de-alta-una-nueva-anotacion-o-propiedad-estandar-de-una-clase}

\subsubsection{Descripci�n breve}

Este caso de uso permite a un usuario a�adir una nueva propiedad de una clase a su grafo. Tan s�lo se podr�n a�adir propiedades entre las previamente configuradas en el c�digo fuente de la aplicaci�n (es decir, han de fijarse de manera previa, si bien su extensi�n es sencilla). La configuraci�n por defecto ofrece una lista de las propiedades m�s utilizadas.

\subsubsection{Actores}

El actor principal es el usuario de la aplicaci�n (fundamentalmente dirigida a un alumno, pero podr�a ser un profesor o cualquier otro usuario).

\subsubsection{Disparadores}

El caso de usa comienza cuando el usuario pulsa en el signo + asociado al nombre de la propiedad a la que quiere dar un valor.

\subsubsection{Flujo de eventos}
\begin{enumerate}
	\item Flujo b�sico
	\begin{enumerate}
		\item El usuario pulsa en el bot�n flotante + asociado a la tarjeta del listado de clases.
		\item Aparece un cuadro de di�logo donde el usuario debe introducir el nombre de la clase.
		\item El usuario introduce el URI del recurso que quiere dar de alta como clase y pulsa en guardar.
		\item El nuevo recurso de tipo clase se crea y aparece en el listado de clases.
	\end{enumerate}
	\item Flujo alternativo 1
	\begin{enumerate}
		\item El usuario pulsa en el bot�n flotante + asociado a la tarjeta del listado de clases.
		\item Aparece un cuadro de di�logo donde el usuario debe introducir el nombre de la clase.
		\item El usuario introduce (o no) el URI del recurso que quiere dar de alta como clase y pulsa fuera del cuadro de di�logo.
		\item El cuadro de di�logo se cierra y no hay ning�n cambio en los recursos.
	\end{enumerate}
	\item Flujo alternativo 2
	\begin{enumerate}
		\item El usuario pulsa en el bot�n flotante + asociado a la tarjeta del listado de clases.
		\item Aparece un cuadro de di�logo donde el usuario debe introducir el nombre de la clase.
		\item El usuario introduce el URI del recurso que quiere dar de alta como clase y pulsa en guardar.
		\item Ocurre un error en el guardado del recurso que le es comunicado al usuario a trav�s de una notificaci�n en pantalla.
	\end{enumerate}
	
\end{enumerate}

\subsubsection{Requisitos especiales}

Ninguno.

\subsubsection{Precondiciones}

El usuario debe haber navegado previamente hasta la secci�n de modelado, bien a trav�s de la pantalla de bienvenida, bien a trav�s de la barra de navegaci�n lateral, y haber seleccionado previamente una clase de entre el listado de clases (en la tarjeta de la izquierda) para poder a�adirle una propiedad.

\subsubsection{Postcondiciones}

Ninguna.

\subsubsection{Extensiones}

Ninguna.

\subsection{Editar una propiedad o anotaci�n de una clase}\label{sec:editar-una-propiedad-o-anotacion-de-una-clase}

\subsubsection{Descripci�n breve}

Este caso de uso permite a un usuario editar el recurso asociado a una propiedad o anotaci�n de una clase existente en su grafo y cambiar su URI. Tan s�lo se podr� editar una propiedad o anotaci�n de entre las previamente configuradas en el c�digo fuente de la aplicaci�n (es decir, han de fijarse de manera previa, si bien su extensi�n es sencilla).

\subsubsection{Actores}

El actor principal es el usuario de la aplicaci�n (fundamentalmente dirigida a un alumno, pero podr�a ser un profesor o cualquier otro usuario).

\subsubsection{Disparadores}

El caso de usa comienza cuando el usuario pulsa en el icono de men� asociado a la propiedad de clase que quiere editar y selecciona la acci�n ``Editar''.

\subsubsection{Flujo de eventos}
\begin{enumerate}
	\item Flujo b�sico
\begin{enumerate}
	\item El usuario pulsa en el icono de men� asociado a la propiedad de la clase que quiere editar y selecciona la acci�n ``Editar''.
	\item Aparece un cuadro de di�logo con el URI del recurso actual que el usuario puede modificar.
	\item El usuario modifica (o no) el URI del recurso  y pulsa en guardar.
	\item El recurso ha sido modificado y su cambio se ve reflejado en el listado de detalle de la propiedad.
\end{enumerate}
\item Flujo alternativo 1
\begin{enumerate}
	\item El usuario pulsa en el icono de men� asociado a la propiedad de la clase que quiere editar y selecciona la acci�n ``Editar''.
	\item Aparece un cuadro de di�logo con el URI del recurso actual que el usuario puede modificar.
	\item El usuario modifica (o no) el URI del recurso  y pulsa fuera del cuadro de di�logo.
	\item El cuadro de di�logo se cierra y no hay ning�n cambio en los recursos.
\end{enumerate}
\item Flujo alternativo 2
\begin{enumerate}
	\item El usuario pulsa en el icono de men� asociado a la propiedad de la clase que quiere editar y selecciona la acci�n ``Editar''.
	\item Aparece un cuadro de di�logo con el URI del recurso actual que el usuario puede modificar.
	\item El usuario modifica (o no) el URI del recurso  y pulsa en guardar.
	\item Ocurre un error en el guardado del recurso que le es comunicado al usuario a trav�s de una notificaci�n en pantalla.
	\end{enumerate}
	
\end{enumerate}

\subsubsection{Requisitos especiales}

Ninguno.

\subsubsection{Precondiciones}

El usuario debe haber navegado previamente hasta la secci�n de modelado, bien a trav�s de la pantalla de bienvenida, bien a trav�s de la barra de navegaci�n lateral, y haber seleccionado previamente una clase de entre el listado de clases (en la tarjeta de la izquierda) para poder editar una de sus propiedades a trav�s del men�.

\subsubsection{Postcondiciones}

Ninguna.

\subsubsection{Extensiones}

Ninguna.

\subsection{Eliminar una propiedad o anotaci�n de una clase}\label{sec:eliminar-una-propiedad-o-anotacion-de-una-clase}

\subsubsection{Descripci�n breve}

Este caso de uso permite a un usuario eliminar el recurso asociado a una propiedad o anotaci�n de una clase existente en su grafo. Tan s�lo se podr� eliminar una propiedad o anotaci�n de entre las previamente configuradas en el c�digo fuente de la aplicaci�n (es decir, han de fijarse de manera previa, si bien su extensi�n es sencilla).

\subsubsection{Actores}

El actor principal es el usuario de la aplicaci�n (fundamentalmente dirigida a un alumno, pero podr�a ser un profesor o cualquier otro usuario).

\subsubsection{Disparadores}

El caso de usa comienza cuando el usuario pulsa en el icono de men� asociado a la propiedad de clase que quiere editar y selecciona la acci�n ``Eliminar''.

\subsubsection{Flujo de eventos}
\begin{enumerate}
	\item Flujo b�sico
	\begin{enumerate}
		\item El usuario pulsa en el icono de men� asociado a la propiedad de la clase que quiere editar y selecciona la acci�n ``Eliminar''.
		\item Aparece un cuadro de di�logo advirtiendo al usuario de que si confirma el cambio, no podr� volver a recuperar el recurso del grafo.
	\item El usuario pulsa fuera del cuadro de di�logo.
	\item El cuadro de di�logo se cierra y no hay ning�n cambio en los recursos.
	\end{enumerate}
	\item Flujo alternativo 1
	\begin{enumerate}
		\item El usuario pulsa en el icono de men� asociado a la propiedad de la clase que quiere editar y selecciona la acci�n ``Eliminar''.
		\item Aparece un cuadro de di�logo advirtiendo al usuario de que si confirma el cambio, no podr� volver a recuperar el recurso del grafo.
	\item El usuario pulsa fuera del cuadro de di�logo.
	\item El cuadro de di�logo se cierra y no hay ning�n cambio en los recursos.
	\end{enumerate}
	\item Flujo alternativo 2
	\begin{enumerate}
		\item El usuario pulsa en el icono de men� asociado a la propiedad de la clase que quiere editar y selecciona la acci�n ``Editar''.
		\item Aparece un cuadro de di�logo advirtiendo al usuario de que si confirma el cambio, no podr� volver a recuperar el recurso del grafo.
	\item El usuario elige cancelar.
	\item El recurso y su detalle siguen apareciendo y no son eliminados.
	\end{enumerate}
	\item Flujo alternativo 3
\begin{enumerate}
	\item El usuario pulsa en el icono de men� asociado a la propiedad de la clase que quiere editar y selecciona la acci�n ``Eliminar''.
	\item Aparece un cuadro de di�logo advirtiendo al usuario de que si confirma el cambio, no podr� volver a recuperar el recurso del grafo.
	\item El usuario elige confirmar.
	\item  Ocurre un error en el borrado del recurso que le es comunicado al usuario a trav�s de una notificaci�n en pantalla.
\end{enumerate}
	
\end{enumerate}

\subsubsection{Requisitos especiales}

Ninguno.

\subsubsection{Precondiciones}

El usuario debe haber navegado previamente hasta la secci�n de modelado, bien a trav�s de la pantalla de bienvenida, bien a trav�s de la barra de navegaci�n lateral, y haber seleccionado previamente una clase de entre el listado de clases (en la tarjeta de la izquierda) para poder editar una de sus propiedades a trav�s del men�.

\subsubsection{Postcondiciones}

Ninguna.

\subsubsection{Extensiones}

Ninguna.

\subsection{Cargar una actividad}\label{sec:cargar-una-actividad}

\subsubsection{Descripci�n breve}

Este caso de uso permite a un usuario cargar en la aplicaci�n un fichero de texto en formato Markdown para su visualizaci�n. Su objetivo es presentar la actividad formativa a llevar a cabo en un formato legible y de forma persistente mientras se lleve a cabo la misma.

\subsubsection{Actores}

El actor principal es el usuario de la aplicaci�n (fundamentalmente dirigida a un alumno, pero podr�a ser un profesor o cualquier otro usuario).

\subsubsection{Disparadores}

El caso de usa comienza cuando el usuario pulsa en el bot�n de ``Importar archivo'' presente en la tarjeta de actividad.

\subsubsection{Flujo de eventos}
\begin{enumerate}
	\item Flujo b�sico
	\begin{enumerate}
		\item El usuario pulsa en el bot�n de ``Importar archivo'' presente en la tarjeta de actividad.
		\item Aparece un cuadro de di�logo para seleccionar un archivo en formato Markdown.
		\item El usuario selecciona un archivo en formato Markdown con extensi�n .md para cargar.
		\item El cuadro de di�logo se cierra y aparece un mensaje informando del �xito en la carga, dem�s de poderse visualizar el contenido de la actividad en la tarjeta de contenido de la misma.
	\end{enumerate}
	\item Flujo alternativo 1
	\begin{enumerate}
		\item El usuario pulsa en el bot�n de ``Importar archivo'' presente en la tarjeta de actividad.
		\item Aparece un cuadro de di�logo para seleccionar un archivo en formato Markdown.
		\item El usuario selecciona un archivo con una extensi�n distinta a .md para cargar.
		\item El cuadro de di�logo se cierra y aparece un mensaje de error informando de que solo es posible cargar archivos en formato Markdown (con extensi�n .md).
	\end{enumerate}
	\item Flujo alternativo 2
	\begin{enumerate}
		\item El usuario pulsa en el bot�n de ``Importar archivo'' presente en la tarjeta de actividad.
		\item El usuario selecciona un archivo en formato Markdown con extensi�n .md para cargar que ocupa m�s del m�ximo permitido (500K, siendo este m�ximo configurable)
		\item El cuadro de di�logo se cierra y aparece un mensaje de error informando de que solo es posible cargar archivos con tama�o inferior al m�ximo permitido, especificando cu�l es dicho m�ximo.
	\end{enumerate}
	
\end{enumerate}

\subsubsection{Requisitos especiales}

Ninguno.

\subsubsection{Precondiciones}

El usuario debe haber navegado previamente hasta la secci�n de Actividades a trav�s de la barra de navegaci�n lateral.

\subsubsection{Postcondiciones}

La actividad se visualizar� con el formato Markdown interpretado en HTML en la tarjeta asociada al contenido de la misma.

\subsubsection{Extensiones}

Ninguna.

\subsection{Cargar un vocabulario}\label{sec:cargar-un-vocabulario}

\subsubsection{Descripci�n breve}

Este caso de uso permite a un usuario cargar en la aplicaci�n los vocabularios m�s comunes de forma predefinida. Este conjunto de vocabularios es f�cilmente extensible a trav�s del c�digo fuente y se carga a trav�s de URLs externas.

\subsubsection{Actores}

El actor principal es el usuario de la aplicaci�n (fundamentalmente dirigida a un alumno, pero podr�a ser un profesor o cualquier otro usuario).

\subsubsection{Disparadores}

El caso de usa comienza cuando el usuario pulsa en uno de los interruptores asociado a cualquiera de los vocabularios m�s utilizados ofrecidos por la aplicaci�n, siempre y cuando dichos interruptores est�n en su posici�n ``desactivado''.

\subsubsection{Flujo de eventos}
\begin{enumerate}
	\item Flujo b�sico
	\begin{enumerate}
		\item El usuario pulsa en un interruptor inactivo de un vocabulario.
		\item El interruptor pasa a estar activo y el vocabulario cargado.
	\end{enumerate}
	\item Flujo alternativo 1
	\begin{enumerate}
		\item El usuario pulsa en un interruptor inactivo de un vocabulario.
		\item Ocurre un error al recuperar o cargar el vocabulario y el interruptor vuelve a su posici�n de desactivado. Simult�neamente, aparece un mensaje de error.
		
	\end{enumerate}
	
\end{enumerate}

\subsubsection{Requisitos especiales}

Es necesario que haya conexi�n a Internet para poder descargar los vocabularios de las URLs precargadas.

\subsubsection{Precondiciones}

El usuario debe haber navegado previamente hasta la secci�n de Vocabularios a trav�s de la barra de navegaci�n lateral.

\subsubsection{Postcondiciones}

Las tripletas de los vocabularios activados estar�n cargadas en el grafo junto con las del modelo.

\subsubsection{Extensiones}

Ninguna.

\subsection{Des-cargar un vocabulario}\label{sec:des-cargar-un-vocabulario}

\subsubsection{Descripci�n breve}

Este caso de uso permite a un usuario des-cargar en la aplicaci�n los vocabularios que haya podido cargar previamente. 

\subsubsection{Actores}

El actor principal es el usuario de la aplicaci�n (fundamentalmente dirigida a un alumno, pero podr�a ser un profesor o cualquier otro usuario).

\subsubsection{Disparadores}

El caso de usa comienza cuando el usuario pulsa en uno de los interruptores asociado a cualquiera de los vocabularios m�s utilizados ofrecidos por la aplicaci�n, siempre y cuando su posici�n est� en ``activado''.

\subsubsection{Flujo de eventos}
\begin{enumerate}
	\item Flujo b�sico
	\begin{enumerate}
		\item El usuario pulsa en un interruptor activo de un vocabulario.
		\item El interruptor pasa a estar inactivo y el vocabulario des-cargado.
	\end{enumerate}
	\item Flujo alternativo 1
	\begin{enumerate}
		\item El usuario pulsa en un interruptor activo de un vocabulario.
		\item Ocurre un error al recuperar o cargar el vocabulario para su borrado del grafo y el interruptor vuelve a su posici�n de ``activado'', puesto que el vocabulario no se ha podido des-cargar. Simult�neamente, aparece un mensaje de error.
		
	\end{enumerate}
	
\end{enumerate}

\subsubsection{Requisitos especiales}

Es necesario que haya conexi�n a Internet para poder descargar los vocabularios de las URLs precargadas y as� eliminarlos del grafo local.

\subsubsection{Precondiciones}

El usuario debe haber navegado previamente hasta la secci�n de Vocabularios a trav�s de la barra de navegaci�n lateral.

\subsubsection{Postcondiciones}

Las tripletas de los vocabularios desactivados ya no estar�n cargadas en el grafo junto con las del modelo.

\subsubsection{Extensiones}

Ninguna.

\subsection{Poblar el grafo con una tripleta}\label{sec:poblar-el-grafo-con-una-tripleta}

\subsubsection{Descripci�n breve}

Este caso de uso permite a un usuario a�adir una tripleta arbitraria al grafo, seleccionando su sujeto (campo libre), predicado (campo de selecci�n que permite introducir nuevos t�rminos) y objeto  (campo de selecci�n que permite introducir nuevos t�rminos).

\subsubsection{Actores}

El actor principal es el usuario de la aplicaci�n (fundamentalmente dirigida a un alumno, pero podr�a ser un profesor o cualquier otro usuario).

\subsubsection{Disparadores}

El caso de usa comienza cuando el usuario comienza a rellenar los campos del formulario de poblamiento y pulsa en aceptar cuando ha terminado. 

\subsubsection{Flujo de eventos}
\begin{enumerate}
	\item Flujo b�sico
	\begin{enumerate}
		\item El usuario rellena los campos del formulario, bien con los selectores, bien con un texto libre.
		\item El usuario pulsa en a�adir instancia.
		\item Aparece un mensaje indicando el �xito de la operaci�n y la instancia se a�ade al grafo.
	\end{enumerate}
	\item Flujo alternativo 1
	\begin{enumerate}
		\item El usuario rellena los campos del formulario pero deja al menos un campo vac�o (sin rellenar).
		\item El usuario pulsa en a�adir instancia.
		\item Aparece un mensaje de error indicando que ning�n elemento de la terna puede ser nulo.
	\end{enumerate}
	
\end{enumerate}

\subsubsection{Requisitos especiales}

Para que las opciones del selector aparezcan precargadas es necesario que, adem�s de los recursos por defecto, se haya cargado alg�n tipo de modelo en el grafo.

\subsubsection{Precondiciones}

El usuario debe haber navegado previamente hasta la secci�n de Poblamiento a trav�s de la barra de navegaci�n lateral.

\subsubsection{Postcondiciones}

La terna a�adida en caso de �xito puede visualizarse utilizando el m�dulo de consultas o, en caso de ser una terna estructural de modelado, en el m�dulo correspondiente.

\subsubsection{Extensiones}

Ninguna.

\subsection{Importar grafo}\label{sec:importar-grafo}

\subsubsection{Descripci�n breve}

Este caso de uso permite a un usuario importar un grafo completo en los formatos N3, Turtle, TriG o N-Triples. El grafo importado sustituir� al cargado por defecto en memoria por la aplicaci�n.

\subsubsection{Actores}

El actor principal es el usuario de la aplicaci�n (fundamentalmente dirigida a un alumno, pero podr�a ser un profesor o cualquier otro usuario).

\subsubsection{Disparadores}

El caso de usa comienza cuando el usuario importa un grafo desde archivo o desde URL.

\subsubsection{Flujo de eventos}
\begin{enumerate}
	\item Flujo b�sico
	\begin{enumerate}
		\item El usuario pulsa en importar desde archivo o rellena la URL del mismo y pulsa en el icono con forma de lupa.
		\item El usuario selecciona el archivo a importar en caso de ser importaci�n desde archivo.
		\item Aparece un mensaje indicando el �xito de la operaci�n y se carga el grafo en el almac�n.
	\end{enumerate}
	\item Flujo alternativo 1
	\begin{enumerate}
		\item  El usuario pulsa en importar desde archivo o rellena la URL del mismo y pulsa en el icono con forma de lupa.
		\item El usuario selecciona o apunta a un archivo que supera el l�mite de tama�o permitido (configurable en la aplicaci�n.)
		\item Aparece un mensaje de error indicando que no es posible cargar el grafo.
	\end{enumerate}
\item Flujo alternativo 2
\begin{enumerate}
	\item El usuario rellena la URL del mismo y pulsa en el icono con forma de lupa.
	\item La URL no est� disponible o devuelve un error de descarga (por ejemplo, CORS, si el servidor no est� preparado).
	\item Aparece un mensaje de error indicando que no es posible cargar el grafo.
\end{enumerate}
	
\end{enumerate}

\subsubsection{Requisitos especiales}

Para cargar archivos desde una URL externa, el servidor remoto tiene que tener correctamente configuradas las cabeceras CORS (\textit{Cross-Origin Resource Sharing})\footnote{\url{https://developer.mozilla.org/es/docs/Web/HTTP/Access_control_CORS}}

\subsubsection{Precondiciones}

El usuario debe haber navegado previamente hasta la secci�n de Importar/Exportar a trav�s de la barra de navegaci�n lateral.

\subsubsection{Postcondiciones}

Una vez cargado el grafo, este podr� visualizarse en el m�dulo de modelado (si tiene ternas estructurales) o en el de consulta en todo caso.

\subsubsection{Extensiones}

Ninguna.

\subsection{A�adir a grafo}\label{sec:anadir-a-grafo}

\subsubsection{Descripci�n breve}

Este caso de uso es una leve modificaci�n del anterior y permite a un usuario a�adir un grafo externo al ya existente en el almac�n en los formatos N3, Turtle, TriG o N-Triples. El grafo importado no sustituir� al cargado por defecto en memoria por la aplicaci�n, sino que pasar� a formar parte de �l.

\subsubsection{Actores}

El actor principal es el usuario de la aplicaci�n (fundamentalmente dirigida a un alumno, pero podr�a ser un profesor o cualquier otro usuario).

\subsubsection{Disparadores}

El caso de usa comienza cuando el usuario a�ade un grafo desde archivo o desde URL.

\subsubsection{Flujo de eventos}
\begin{enumerate}
	\item Flujo b�sico
	\begin{enumerate}
		\item El usuario pulsa en a�adir desde archivo o rellena la URL del mismo y pulsa en el icono con forma de lupa.
		\item El usuario selecciona el archivo a a�adir en caso de ser importaci�n desde archivo.
		\item Aparece un mensaje indicando el �xito de la operaci�n y se a�ade el grafo en el almac�n.
	\end{enumerate}
	\item Flujo alternativo 1
	\begin{enumerate}
		\item  El usuario pulsa en a�adir a archivo o rellena la URL del mismo y pulsa en el icono con forma de lupa.
		\item El usuario selecciona o apunta a un archivo que supera el l�mite de tama�o permitido (configurable en la aplicaci�n.)
		\item Aparece un mensaje de error indicando que no es posible cargar el grafo.
	\end{enumerate}
	\item Flujo alternativo 2
	\begin{enumerate}
		\item El usuario rellena la URL del mismo y pulsa en el icono con forma de lupa.
		\item La URL no est� disponible o devuelve un error de descarga (por ejemplo, CORS, si el servidor no est� preparado).
		\item Aparece un mensaje de error indicando que no es posible a�adir el grafo.
	\end{enumerate}
	
\end{enumerate}

\subsubsection{Requisitos especiales}

Para cargar archivos desde una URL externa, el servidor remoto tiene que tener correctamente configuradas las cabeceras CORS (\textit{Cross-Origin Resource Sharing})\footnote{\url{https://developer.mozilla.org/es/docs/Web/HTTP/Access_control_CORS}}

\subsubsection{Precondiciones}

El usuario debe haber navegado previamente hasta la secci�n de Importar/Exportar a trav�s de la barra de navegaci�n lateral.

\subsubsection{Postcondiciones}

Una vez a�adido el grafo, este podr� visualizarse junto con el existente previamente en el m�dulo de modelado (si tiene ternas estructurales) o en el de consulta en todo caso.

\subsubsection{Extensiones}

Ninguna.

\subsection{Exportar grafo}\label{sec:exportar-grafo}

\subsubsection{Descripci�n breve}

Este caso de uso  permite a un usuario exportar a un archivo de texto en formato Turtle o JSON-LD el grafo de trabajo.

\subsubsection{Actores}

El actor principal es el usuario de la aplicaci�n (fundamentalmente dirigida a un alumno, pero podr�a ser un profesor o cualquier otro usuario).

\subsubsection{Disparadores}

El caso de usa comienza cuando el usuario pulsa en el bot�n de ``Exportar JSON-LD'' o ``Exportar Turtle''.

\subsubsection{Flujo de eventos}
\begin{enumerate}
	\item Flujo b�sico
	\begin{enumerate}
		\item El usuario pulsa en el bot�n de ``Exportar JSON-LD'' o ``Exportar Turtle''.
		\item El grafo se exporta y aparece un cuadro de di�logo para guardar el archivo.
	\end{enumerate}
	\item Flujo alternativo 1
	\begin{enumerate}
		\item El usuario pulsa en el bot�n de ``Exportar JSON-LD'' o ``Exportar Turtle''.
		\item Ocurre un error inesperado que se muestra al usuario.
	\end{enumerate}
	
\end{enumerate}

\subsubsection{Requisitos especiales}

Ninguno.

\subsubsection{Precondiciones}

Debe existir un grafo cargado en memoria.

\subsubsection{Postcondiciones}

Ninguna.

\subsubsection{Extensiones}

Ninguna.

\subsection{Lanzar consulta a grafo local}\label{sec:lanzar-consulta-a-grafo-local}

\subsubsection{Descripci�n breve}

Este caso de uso  permite a un usuario lanzar una consulta SPARQL al grafo local cargado en memoria, bien desde un archivo con extensi�n ``rq'' o URL, bien desde un cuadro de texto.

\subsubsection{Actores}

El actor principal es el usuario de la aplicaci�n (fundamentalmente dirigida a un alumno, pero podr�a ser un profesor o cualquier otro usuario).

\subsubsection{Disparadores}

El caso de usa comienza cuando el usuario carga la consulta pulsa en el bot�n de ``Lanzar Consulta''.

\subsubsection{Flujo de eventos}
\begin{enumerate}
	\item Flujo b�sico
	\begin{enumerate}
		\item El usuario carga la consulta a lanzar, bien escribi�ndola en el cuadro de texto, bien carg�ndola desde un archivo ``rq'' o URL donde se encuentre.
		\item El usuario pulsa en el bot�n de ``Lanzar Consulta''.
		\item Los resultados de la consulta aparecen en la tabla de ``Resultados''.
		
	\end{enumerate}
	\item Flujo alternativo 1
	\begin{enumerate}
			\item El usuario carga la consulta a lanzar, bien escribi�ndola en el cuadro de texto, bien carg�ndola desde un archivo ``rq'' o URL donde se encuentre.
		\item El usuario pulsa en el bot�n de ``Lanzar Consulta''.
		\item Ocurre un error inesperado o de sintaxis al lanzar la consulta y aparece un mensaje de error para el usuario.
	\end{enumerate}
	
\end{enumerate}

\subsubsection{Requisitos especiales}

Ninguno.

\subsubsection{Precondiciones}

Debe existir un grafo cargado en memoria.

\subsubsection{Postcondiciones}

Ninguna.

\subsubsection{Extensiones}

Ninguna.

\subsection{Lanzar consulta a grafo remoto}\label{sec:lanzar-consulta-a-grafo-remoto}

\subsubsection{Descripci�n breve}

Este caso de uso  permite a un usuario lanzar una consulta SPARQL a un grafo ubicado en un endpoint remoto, bien desde un archivo con extensi�n ``rq'' o URL, bien desde un cuadro de texto.

\subsubsection{Actores}

El actor principal es el usuario de la aplicaci�n (fundamentalmente dirigida a un alumno, pero podr�a ser un profesor o cualquier otro usuario).

\subsubsection{Disparadores}

El caso de usa comienza cuando el usuario activa la opci�n ``Usar endpoint SPARQL remoto''.

\subsubsection{Flujo de eventos}
\begin{enumerate}
	\item Flujo b�sico
	\begin{enumerate}
		\item El usuario activa la opci�n ``Usar endpoint SPARQL remoto''.
		\item El usuario introduce la URL del endpoint remoto.
		\item El usuario carga la consulta a lanzar, bien escribi�ndola en el cuadro de texto, bien carg�ndola desde un archivo ``rq'' o URL donde se encuentre.
		\item El usuario pulsa en el bot�n de ``Lanzar Consulta''.
		\item Los resultados de la consulta al grafo remoto aparecen en la tabla de ``Resultados''.
		
	\end{enumerate}
	\item Flujo alternativo 1
	\begin{enumerate}
		\item El usuario activa la opci�n ``Usar endpoint SPARQL remoto''.
		\item El usuario introduce la URL del endpoint remoto.
		\item El usuario carga la consulta a lanzar, bien escribi�ndola en el cuadro de texto, bien carg�ndola desde un archivo ``rq'' o URL donde se encuentre.
		\item El usuario pulsa en el bot�n de ``Lanzar Consulta''.
		\item Ocurre un error inesperado, de URL o de sintaxis al lanzar la consulta y aparece un mensaje de error para el usuario.
	\end{enumerate}
	
\end{enumerate}

\subsubsection{Requisitos especiales}

Ninguno.

\subsubsection{Precondiciones}

Los endpoints remotos deben soportar el protocolo SPARQL y tener correctamente configuradas las cabeceras CORS.

\subsubsection{Postcondiciones}

Ninguna.

\subsubsection{Extensiones}

Ninguna.\cleardoublepage{}

\chapter{Dise�o}

\section{Arquitectura de una aplicaci�n sem�ntica}

Independientemente de la tecnolog�a elegida para su implementaci�n, una aplicaci�n sem�ntica gen�rica se caracteriza por una arquitectura concreta que se va a proceder a detallar. Dicha arquitectura est� basada en componentes que pueden ser proporcionados por uno o varios productos de mercado. Es com�n encontrar los siguientes:

\begin{itemize}  
	\item \textbf{Almac�n de ternas} (\textit{store}): no es m�s que una base de datos que permite el almacenamiento y extracci�n de ternas RDF, as� como consolidar informaci�n procedente de distintas fuentes de datos.
	\item \textbf{Procesador/Serializador} (\textit{parser/serializer}): un procesador RDF lee ficheros de texto y los interpreta como ternas RDF; un serializador realiza el proceso inverso.
	\item \textbf{Motor de consultas} (\textit{query engine}): provee a la aplicaci�n de la funcionalidad de recuperaci�n de informaci�n en base a consultas estructuradas.
\end{itemize}

A continuaci�n se estudian con m�s detalle, ofreciendo ejemplos de los m�s utilizados y el caso concreto del proyecto que ata�e a esta memoria.

\subsection{Almac�n RDF (\textit{store})}

El almac�n de ternas es un componente esencial para salvaguardar conjuntos de datos en ternas independientemente de su origen (un fichero serializado, el resultado de una consulta, etc.). Conceptualmente, el equivalente en una base de datos relacional ser�a una tabla con tres columnas (Sujeto - Predicado - Objeto). Sin embargo, parece evidente que un almac�n RDF est� optimizado para el almacenamiento y recuperaci�n de ternas RDF.

Comparados con una base de datos relacional, sin embargo, son m�s flexibles y requieren de menor coste de uso y mantenimiento. M�s concretamente:

\begin{itemize}  
	\item \textbf{Flexibilidad:} su esquema flexible permite realizar cambios sin paradas ni redise�os, cosa que no ocurre con una base de datos relacional.
	\item \textbf{Estandarizaci�n:} el nivel de estandarizaci�n de RDF y SPARQL es mucho mayor que el de SQL. Es pr�cticamente inmediato sustituir un almac�n de tripletas por otro, mientras que en el caso de las bases de datos relacionales es necesario tener en cuenta los distintos dialectos e implementaciones de SQL seg�n el fabricante.
	\item \textbf{Expresividad:} es mucho m�s sencillo modelar datos complejos y con elevado n�mero de relaciones entre ellos en RDF que en SQL. Ocurre al contrario si los datos son de naturaleza tabular, naturalmente.
\end{itemize}

No todo son ventajas en la comparaci�n:

\begin{itemize}  
	\item \textbf{Madurez:} las bases de datos relacionales son mucho m�s maduras y presentan m�s funcionalidades que los almacenes de ternas.
	\item \textbf{Coste de almacenamiento:} El coste por unidad de informaci�n almacenada en un almac�n de ternas es mucho mayor que el de una base de datos relacional, lo cual puede ser cr�tico si se est�n tratando grandes vol�menes de datos.
\end{itemize}

Entre los almacenes de ternas m�s utilizados, se pueden citar OpenLink Virtuoso\footnote{https://virtuoso.openlinksw.com/}, GraphDB de Ontotext\footnote{https://ontotext.com/products/graphdb/}, Apache Jena TDB\footnote{https://jena.apache.org/documentation/tdb/} o AllegroGraph\footnote{https://franz.com/agraph/allegrograph/}.


\subsubsection{Almac�n en el proyecto}

En el caso del presente proyecto, se valoraron dos alternativas:

\begin{itemize}  
	\item \textbf{RDF-Ext Dataset\footnote{https://github.com/rdf-ext/rdf-store-dataset}:} un simple conjunto de datos cargado en memoria y conforme con la especificaci�n est�ndar. Este almac�n presentaba limitaciones en cuanto a rendimiento y n�mero de tripletas a almacenar, adem�s de estar pensado para operaciones s�ncronas.
	\item \textbf{N3.js:\footnote{https://github.com/rdfjs/N3.js/}} La bilioteca N3 ofrece, adem�s de almacenamiento de tripletas en memoria en Javascript nativo, facilidades de serializaci�n y procesado con formatos est�ndar como Turtle, N3, N-Triples y TriG. Si bien su conformidad con la especificaci�n de interfaces RDF.js no es completa, s� est� presente en gran parte de sus m�dulos (\textit{DataFactory, StreamParser, StreamWriter y Store}).
\end{itemize}

Confirmadas las limitaciones de RDF-Ext Dataset y consultado con el \textit{W3C RDFJS Community Group}, se opta finalmente por utilizar \textbf{N3.js} (lo que supuso adaptar ligeramente la implementaci�n del proyecto al nuevo \textit{storage}).


\subsection{Procesador y Serializador RDF (\textit{parser/serializer})}

En muchos casos, los m�dulos para importaci�n y exportaci�n de datos en RDF son proporcionados por el propio almac�n. Sin entrar a valorar su sintaxis dada su complejidad espec�fica y por considerarse fuera del inter�s de este proyecto, s� se va a proceder a identificar y caracterizar los formatos de serializaci�n m�s comunes.

\subsubsection{RDF/XML}

Se trata de una representaci�n en XML, com�nmente criticada por su dificultad de lectura por parte de personas. Entre sus otras cr�ticas est�n las limitaciones impuestas por las reglas de nomenclatura de XML o los problemas encontrados para procesar este formato con herramientas populares para XML\cite{ducharme01}. Para poder codificar un grafo RDF en XML, tanto los nodos como los predicados han de ser representados en t�rminos de XML (elementos, atributos, contenido de elementos y valores de atributos). 

Conceptualmente, se construye a partir de un conjunto de descripciones m�s peque�as, cada una de las cuales traza un camino a lo largo de un grafo RDF. Estos caminos se describen en t�rminos de los nodos (sujetos) y enlaces (predicados), que los conectan con otros nodos (objetos).

A continuaci�n se muestra un ejemplo de serializaci�n en RDF/XML\footnote{https://www.w3.org/TR/rdf-syntax-grammar/\#example3}:

\begin{lstlisting}[basicstyle=\tiny, language=XML, caption={Ejemplo de RDF/XML del W3C.},captionpos=b]

<rdf:Description rdf:about="http://www.w3.org/TR/rdf-syntax-grammar">
<ex:editor>
<rdf:Description>
<ex:homePage>
<rdf:Description rdf:about="http://purl.org/net/dajobe/">
</rdf:Description>
</ex:homePage>
</rdf:Description>
</ex:editor>
</rdf:Description>

<rdf:Description rdf:about="http://www.w3.org/TR/rdf-syntax-grammar">
<ex:editor>
<rdf:Description>
<ex:fullName>Dave Beckett</ex:fullName>
</rdf:Description>
</ex:editor>
</rdf:Description>

<rdf:Description rdf:about="http://www.w3.org/TR/rdf-syntax-grammar">
<dc:title>RDF 1.1 XML Syntax</dc:title>
</rdf:Description>

\end{lstlisting}

\subsubsection{N-Triples}

Se trata de una notaci�n muy simple pero verbosa, donde cada l�nea representa una �nica declaraci�n conteniendo sujeto, predicado y objeto seguidos por un punto.

Ejemplo\footnote{https://www.w3.org/TR/n-triples/}:

\begin{lstlisting}[basicstyle=\tiny, language=XML, caption={Ejemplo de N-Triples del W3C.},captionpos=b]

<http://example.org/#spiderman> <http://www.perceive.net/schemas/relationship/enemyOf> <http://example.org/#green-goblin> .

\end{lstlisting}

\subsubsection{N3}

N3, abreviatura de \textit{Notation 3}, fue un proyecto personal de Tim Berners-Lee\cite{ducharme01}. Es muy similar a N-Triples, pero a�adiendo caracter�sticas adicionales como atajos, un formato m�s claro, o la condensaci�n de muchas de las repeticiones de este formato.

A pesar de todo, N3 nunca se convirti� en un est�ndar y sus caracter�sticas adicionales frente a N-Triples no tuvieron mucha aceptaci�n.

\subsubsection{Turtle}

El formato Turtle permite escribir un grafo RDF en texto de una forma m�s compacta que RDF/XML y m�s sencilla de leer que N-Triples. Su gram�tica es un subconjunto de la especificaci�n del lenguaje de consulta SPARQL 1.1, compartiendo ambas nombres de terminales y producciones en la medida de lo posible.

En estos momentos, es el formato de serializaci�n m�s popular entre las comunidades de desarrolladores\cite{yu01}.

A continuaci�n se muestra un ejemplo de serializaci�n en Turtle\footnote{https://www.w3.org/TR/turtle/}:

\begin{lstlisting}[basicstyle=\tiny, language=XML, caption={Ejemplo de Turtle del W3C.},captionpos=b]

@base <http://example.org/> .
@prefix rdf: <http://www.w3.org/1999/02/22-rdf-syntax-ns#> .
@prefix rdfs: <http://www.w3.org/2000/01/rdf-schema#> .
@prefix foaf: <http://xmlns.com/foaf/0.1/> .
@prefix rel: <http://www.perceive.net/schemas/relationship/> .

<#green-goblin>
rel:enemyOf <#spiderman> ;
a foaf:Person ;    # in the context of the Marvel universe
foaf:name "Green Goblin" .

<#spiderman>
rel:enemyOf <#green-goblin> ;
a foaf:Person ;
foaf:name "Spiderman" .

\end{lstlisting}

\subsubsection{Procesadores y serializadores en el proyecto}

El almac�n seleccionado, \textbf{N3.js, proporciona procesadores y serializadores} para los principales formatos citados: N3, N-Triples, Turtle y TriG (una extensi�n de este �ltimo).

No se ha considerado necesario dise�ar ning�n serializador o procesador personalizado, al estar ya incluidos los formatos m�s comunes utilizados por desarrolladores (n�tese la ausencia deliberada de RDF/XML).


\subsection{Motor de consulta RDF (\textit{query engine})}

El motor de consulta es un componente �ntimamente ligado al almac�n de ternas. El W3C fij� un est�ndar para consultas: SPARQL (presentado en \ref{sparql}), cuya especificaci�n est� en su versi�n 1.1 desde 2013.

Desde el punto de vista de la arquitectura de una aplicaci�n sem�ntica, es deseable que el motor de consultas, adem�s de cumplir con la especificaci�n, est� �ntimamente integrado con el almac�n de ternas para ofrecer un rendimiento aceptable.

Los principales proveedores de almacenes de ternas ofrecen tambi�n motores de consulta integrados, como Apache Jena, Virtuoso, AllegroGraph...


\subsubsection{Motor de consultas en el proyecto}

En el caso del presente proyecto, integrar un motor de consultas no fue sencillo. El componente N3.js no ofrece dichas funcionalidades, con lo que fue necesario estudiar la viabilidad de incorporar un motor local totalmente desarrollado en Javascript.

Tras el estudio de los trabajos previos desarrollados (\ref{subsection:consulta}) y la  situaci�n actual (\ref{subsubsection:manejosparql}), se lleg� a la conclusi�n de que la plataforma Comunica se posicionaba como el producto abierto m�s prometedor para satisfacer las necesidades de consulta SPARQL.

Comunica proporciona las herramientas necesarias para crear una aplicaci�n combinando m�ltiples bloques de construcci�n independientes. Su prop�sito es ofrecer una implementaci�n modular de un cliente \textit{Triple Pattern Fragment} o\textit{ Linked Data Fragment}, un tipo de fragmento que consiste en:

\begin{itemize}  
	\item \textbf{Datos} que corresponden a un patr�n de tripleta.
	\item \textbf{Metadatos} que consisten en el total aproximado de tripletas.
	\item \textbf{Controles} que llevan al resto de fragmentos del mismo conjunto de datos.
\end{itemize}

De forma resumida, un cliente que soporta este tipo de fragmentos puede resolver m�ltiples consultas SPARQL de forma eficiente.

Comunica consta de los siguientes componentes:

\begin{itemize}  
	\item \textbf{Actores:} definen el formato de la entrada que aceptan y la salida correspondiente que producen.
	\item \textbf{Buses:} combinan los actores que soportan el mismo formato de entrada y salida y permite enviar mensajes a todos los actores registrados en un bus dado.
	\item \textbf{Mediadores:} envuelven los buses y se aseguran de que cada petici�n recibe tan s�lo una respuesta.
\end{itemize}

\begin{figure}[h]
	\centering
	\includesvg[width = 500pt]{actor-mediator-bus}
	\caption{Arquitectura de Comunica}
\end{figure}

Para el presente proyecto, han sido necesarios dos actores:

\begin{itemize}  
	\item \textbf{actor-init-sparql:} se trata de un cliente que permite resolver consultas sobre interfaces heterog�neos. En concreto, este m�dulo inicializa un motor Comunica con actores que eval�an consultas SPARQL.
	\item \textbf{actor-init-sparql-rdfjs:} permite lanzar consultas SPARQL a fuentes que implementan la interfaz \textit{Source}\footnote{http://rdf.js.org/\#source-interface}. Una de esas fuentes es N3.js, el \textit{store} utilizado para almacenar ternas en el proyecto.
\end{itemize}

Es decir, se utiliza un m�dulo para lanzar consultas a \textit{endpoints} SPARQL remotos y otro para la fuente o almac�n local. Estos m�dulos \textbf{a�n est�n en desarrollo y no han alcanzado una versi�n estable} a fecha de redacci�n de la presente memoria, con lo que se van adquiriendo nuevas funcionalidades y corrigiendo defectos a lo largo del tiempo de desarrollo de este proyecto (por ejemplo, en Julio de 2018 el motor de consulta no soportaba la palabra clave SERVICE, caracter�stica implementada en Agosto).

\section{Arquitectura de la aplicaci�n Vue}

Al margen de su naturaleza sem�ntica, la herramienta desarrollada se implementa en Javascript ES6+ y Vue.js. Es necesario, por tanto, proponer un dise�o de alto nivel de sus componentes.

\subsection{Almacenamiento de estado}

Para almacenar el estado de la aplicaci�n se ha optado por usar Vuex\footnote{https://vuex.vuejs.org/}. 

Vuex es una biblioteca que implementa un patr�n de gesti�n de estado (ver figura \ref{fig:vuex}\cite{vuex01}) basado en Redux\footnote{Redux es un contenedor predecible del estado de aplicaciones Javascript. Ver https://es.redux.js.org/}, Flux\footnote{Una arquitectura de aplicaci�n para React que utiliza un flujo de datos unidireccional. Ver https://github.com/facebook/flux} y la arquitectura Elm\footnote{Un patr�n sencillo de arquitectura de aplicaciones web. Ver https://guide.elm-lang.org/architecture/}. Se utiliza como un almac�n centralizado para todos los componentes de la aplicaci�n, y permite asegurar:

\begin{itemize}  
	\item Que todos los componentes comparten ese estado.
	\item Que el estado s�lo podr� ser modificado de forma controlada, bien s�ncrona o as�ncrona.
\end{itemize}

\begin{figure}[h]
	\centering
	\includegraphics[scale=0.5]{vuex}
	\caption{Vuex: gesti�n de estado centralizado.}
	\label{fig:vuex}
\end{figure}

Sin entrar en detalles, el funcionamiento de Vuex se ci�e a las siguientes reglas:

\begin{itemize}  
	\item Existe un �nico estado compartido por todos los componentes de la aplicaci�n, independientemente de que cada componente tenga su propio estado individual.
	\item El acceso al estado centralizado se realiza a trav�s de \textbf{getters}.
	\item El estado s�lo puede ser modificado a trav�s de \textbf{mutaciones s�ncronas}.
	\item Las mutaciones pueden ser ejecutadas de manera as�ncrona a trav�s de \textbf{acciones}.
\end{itemize}

\subsection{Arquitectura de componentes}

Para poder dar cumplimiento a los requisitos de la aplicaci�n, se definen los siguientes componentes con sus �mbitos de responsabilidad:

\vspace*{\baselineskip}
\begin{center}
	\begin{table}[htb]
		\centering
		\begin{tabular}{p{1cm}p{3cm}p{10cm}}
			\toprule
			ID & Componente & Responsabilidad \\ \midrule
			1 & Contenedor de aplicaci�n &  Estructura b�sica de la aplicaci�n (cabecera, pie, barra lateral). \\ \midrule
			2 & Bienvenida &  Entrada a la aplicaci�n con ayuda y descripci�n general. \\ \midrule
			3 & Contenedor de modelo &  Contiene los componentes de edici�n del modelo.\\ \midrule
			4 & Contenedor de importaci�n/exportaci�n & Contiene los componentes de importaci�n y exportaci�n de datos. \\ \midrule
			5 & Contenedor de consultas SPARQL &  Contiene los componentes de ejecuci�n y visualizaci�n de resultados de consultas SPARQL. \\ \midrule
			6 & Contenedor de actividades &  Contiene los componentes de carga y visualizaci�n de actividades. \\ \midrule
			7 & Listado de recursos &  Componente reutilizable para listar recursos en una tabla de datos. \\ \midrule
			8 & Detalle de recursos &  Componente reutilizable para visualizar detalles de recursos. \\ \midrule
			9 & Lanzador de consultas SPARQL & Componente reutilizable para lanzar consultas SPARQL \\ \midrule
			10 & Listado de resultados de consultas SPARQL & Componente reutilizable para mostrar resultados de una consulta SPARQL.\\ \midrule
			11 & Cargador de archivos &  Componente reutilizable para importar datos desde archivo. \\ \midrule
			12 & Cargador de URLs &  Componente reutilizable para importar datos desde URL. \\ \midrule
			13 & Cargador de Vocabularios & Componente reutilizable para habilitar o deshabilitar vocabularios. \\ \midrule
			14 & Visor de markdown & Componente reutilizable para visualizar markdown renderizado en HTML. \\ 
			\bottomrule
		\end{tabular}
		\caption{Listado de componentes de la aplicaci�n.}
		\label{tab:componentes}
	\end{table}
\end{center}\cleardoublepage{}

\chapter{Implementaci�n y pruebas}

\section{Implementaci�n}

\subsection{Est�ndar Javascript}

Desde su nacimiento en 1995 de la mano de Netscape como lenguaje de \textit{scripting} para mejorar la experiencia final del usuario de Internet a trav�s de un navegador, Javascript ha evolucionado hasta convertirse en la tecnolog�a dominante en el desarrollo de frontend web. 

Es importante describir dicha evoluci�n a trav�s de sus distintos est�ndares (ECMAScript) para contextualizar el presente desarrollo:

	\begin{itemize}
	\item ES5, publicado en 2009 y soportado por la pr�ctica totalidad de los navegadores (incluidos los antiguos).
	\item ES2015, tambi�n conocido como ES6,fue la primera de una serie de revisiones del est�ndar con car�cter anual, despu�s de varios a�os transcurridos desde el est�ndar anterior. Se trata de una versi�n rompedora con los est�ndares anteriores, que incorpora importantes novedades de sintaxis.
	\item ES2016 y posteriores, que van incorporando m�s mejoras como la gesti�n de eventos as�ncronos con \textit{async/await}.
\end{itemize}

Aunque los navegadores est�n haciendo grandes progresos con su compatibilidad total con ES6, a�n no es recomendable utilizarlo en forma nativa. Para resolver este problema surgen los transpiladores o \textit{transpilers} como Babel\footnote{\url{https://babeljs.io/}} (utilizado en este proyecto), que traducen el lenguaje programado en un est�ndar m�s moderno como ES2015+ a ``Vanilla ES5'', de tal manera que el c�digo pueda ser ejecutado en cualquier navegador.

A la vista de lo expuesto, se plantea la implementaci�n de la presente aplicaci�n en ES6+ (es decir, ES6 y posteriores), pero efectuando un \textit{transpiling} a ES5 para minimizar cualquier problema de compatibilidad con los distintos navegadores.

\subsection{Entorno}

Hoy en d�a, no es posible hablar de desarrollo profesional y moderno en Javascript sin hablar de Node.js\footnote{\url{https://nodejs.org}}. Node.js es un entorno de ejecuci�n para Javascript construido con el motor V8 de Chrome\footnote{El motor de alto rendimiento de Google para Javascript escrito en C++. Ver \url{https://developers.google.com/v8/}} y orientado a eventos as�ncronos. Est� dise�ado para construir aplicaciones de red escalables. Introdujo un sistema de m�dulos que ha sido la base de la reutilizaci�n de bibliotecas de c�digo en Javascript a trav�s del repositorio npm\footnote{https://docs.npmjs.com/getting-started/what-is-npm}, con en torno a 600.000 paquetes de c�digo abierto disponibles para la comunidad de desarrolladores.

En la implementaci�n de esta aplicaci�n se han utilizado las siguientes versiones de estos componentes, cuya instalaci�n previa es necesaria para la construcci�n y ejecuci�n de la misma:

	\begin{itemize}
	\item Node.js 10.7.0
	\item npm 6.1.0

\end{itemize}

Tanto Node como npm pueden descargarse e instalarse en m�ltiples sistemas operativos desde sus respectivas p�ginas web.

\subsection{Estructura del proyecto}

El proyecto presenta la siguiente estructura jer�rquica de directorios:

\begin{itemize}
	\item \textbf{build}: contiene los archivos de configuraci�n de Webpack, tanto para el servidor de desarrollo como para el entorno de producci�n.
	\item \textbf{config}: contiene los archivos de configuraci�n del proyecto por entorno (dev, test, pro). Si bien no es conveniente modificarlo, para la presente aplicaci�n tuvo que a�adirse el soporte para babel-polyfill para permitir la correcta ejecuci�n de determinados paquetes en todos los navegadores.
	\item \textbf{dist}: contiene los archivos que deber�n ser desplegados en la plataforma de producci�n (por ejemplo, Firebase).
	\item \textbf{docs}: notas del proyecto en markdown y c�digo fuente en \LaTeX de esta memoria.
	\item \textbf{node\_modules}: contiene todos los paquetes y dependencias de npm necesarios para ejecutar el proyecto.
	\item \textbf{src}: contiene el c�digo fuente de la aplicaci�n y est� estructurado, a su vez, de la siguiente forma:
	\begin{itemize}
		\item \textbf{components}: contiene los distintos componentes de la aplicaci�n, siendo App.vue el principal.
		\item \textbf{fonts}: contiene las fuentes necesarias para visualizar correctamente Material Design.
		\item \textbf{router}: contiene la configuraci�n de vue-router, un enrutador para Vue.js.
		\item \textbf{services}: contiene la configuraci�n �nica de servicios para recuperar informaci�n de APIs REST a trab�s dfe bibliotecas como axios\footnote{Un cliente HTTP basado en promesas para Node.js. Ver \url{https://github.com/axios/axios}}
		\item \textbf{store}: contiene la l�gica necesaria para mantener el estado �nico de la aplicaci�n a trab�s de Vuex (ficheros de acciones, mutaciones, getters y el propio almac�n.)
		\item \textbf{utils}: contiene utilidades o informaci�n por defecto necesaria para el correcto funcionamiento de la aplicaci�n.
	\end{itemize}
	\item \textbf{static}: recursos est�ticos que son copiados directamente.
	\item \textbf{test}: contiene la configuraci�n y especificaciones de las pruebas unitarias y extremo a extremo.
	
\end{itemize}

Adem�s, en el directorio ra�z de la estructura del proyecto se pueden encontrar, entre otros, los siguientes archivos de configuraci�n relevantes:

\begin{itemize}
	\item \textbf{.babelrc}: configuraci�n del \textit{transpiler} Babel.
	\item \textbf{.gitignore}: archivos de proyecto que no deber�n subirse al repositorio (binarios, temporales, etc.)
	\item \textbf{.eslintrc.js}: configuraci�n del analizador est�tico de estilo y sintaxis ESlint.
	\item \textbf{package.json}: contiene la configuraci�n de la tarea de construcci�n y la especificaci�n de dependencias del proyecto.
	\item \textbf{.firebaserc}: configuraci�n de proyecto de Firebase
	\item \textbf{firebase.json}: configuraci�n de despliegue en Firebase (ver subsecci�n \ref{sec:despliegue-del-proyecto}).
	
\end{itemize}

\subsection{Construcci�n del proyecto}

El proyecto de trabajo se ha generado usando la herramienta vue-cli\footnote{\url{https://cli.vuejs.org/}} en su versi�n 2 (durante el tiempo de desarrollo de este proyecto se public� la versi�n 3). Esta herramienta permite generar proyectos configurados a partir de una plantilla y selecci�n de par�metros. En concreto, se utiliz� una plantilla propuesta por Vuetify\footnote{\url{https://vuetifyjs.com}}, una biblioteca de componentes de interfaz de usuario basada en Material Design\footnote{Un sistema de dise�o para desarrollar interfaces de usuario impulsado por Google. Ver \url{https://material.io/}} para Vue. Se eligi� esta biblioteca por ser una de las m�s usadas entre la comunidad de usuarios, disponer de un amplio cat�logo de componentes implementados y presentar l�neas de trabajo prometedoras para finales de este a�o (con la previsi�n de publicaci�n de nuevos componentes y de mejora de los actuales).

La propia web de Vuetify presenta un asistente para elegir la plantilla m�s adecuada bas�ndose en diversos par�metros introducidos por el usuario, tales como: experiencia con Vue, tipo de interfaz (web, m�vil, escritorio), necesidad o no de generar estructuras de pruebas unitarias, etc. Dadas las necesidades de este proyecto, se opt� por elegir una plantilla para programadores con poca experiencia en Vue, para la web y con estructuras de pruebas unitarias (y no con posicionamiento web). 

El resultado fue la plantilla webpack\footnote{Webpack es un empaquetador de m�dulos para JS. Ver \url{https://github.com/vuetifyjs/webpack} y \url{https://webpack.js.org/}}, que presenta las siguientes caracter�sticas:

\begin{itemize}
	\item Desarrollo:
	\begin{itemize}
		\item Incluye Webpack y vue-loader para SFC (\textit{Single File Components})
		\item Recarga en caliente con preservaci�n de estado
		\item An�lisis est�tico de estilo con ESlint\footnote{\url{https://eslint.org/}} al guardar
	\end{itemize}
	\item Construcci�n:
		\begin{itemize}
		\item Minificaci�n de JS con UglifyJS\footnote{\url{https://github.com/mishoo/UglifyJS}}
		\item Minificaci�n de HTML con html-minifier
		\item Minificaci�n de CSS con cssnano
		\item Compilaci�n de recursos est�ticos con hashes de versiones para cacheo eficiente
		\end{itemize}
	\item Pruebas:
		\begin{itemize}
		\item Soporte de ES2015+ en archivos de test
		\item Jest\footnote{\url{https://jestjs.io/}} como plataforma de testing
		\item End-to-end\footnote{Las pruebas end-to-end o extremo a extremo permiten verificar el flujo de trabajo completo de una aplicaci�n web. Son m�s lentos que los unitarios y dif�ciles de depurar, pero comprueban la funcionalidad completa de la aplicaci�n.} con Nightwatch\footnote{\url{http://nightwatchjs.org/}}.
	\end{itemize}
\end{itemize}

\subsection{Ejecuci�n del proyecto}

De manera predeterminada, es posible ejecutar los siguientes comandos:

	\begin{itemize}
	\item \textbf{npm install}: instala todas las dependencias del proyecto.
	\item \textbf{npm run dev}: sirve el contenido con recarga en caliente a trav�s del puerto 8080.
	\item \textbf{npm run build}: construye el proyecto para desplegar en producci�n (con compresi�n).
	\item \textbf{npm run build --report}: construye el proyecto para producci�n y visualiza el informe del analizador de empaquetado.
	\item \textbf{npm run unit}: ejecuta las pruebas unitarias.
	\item \textbf{npm run e2e}: ejecuta las pruebas end-to-end.
	\item \textbf{npm test}: ejecuta todas las pruebas.
	
\end{itemize}


\subsection{Despliegue del proyecto}\label{sec:despliegue-del-proyecto}

Para desplegar la aplicaci�n a modo de demostraci�n se ha optado por utilizar Firebase\footnote{\url{https://firebase.google.com/}}, una plataforma de desarrollo de Google, a modo de PaaS. Si bien Firebase provee de muchos servicios en la nube (como base de datos, autenticaci�n, almacenamiento, etc.) para este proyecto tan solo ha sido necesario utilizar el servicio de hosting. Para ello, basta con incluir un par de sencillos ficheros de configuraci�n con el siguiente contenido:

\begin{lstlisting}[caption={Configuraci�n de Firebase},captionpos=b]

{
	"hosting": {
		"public": "./dist",
		"ignore": [
			"firebase.json",
			"**/.*",
			"**/node_modules/**"
		]
	}
}
\end{lstlisting} 

\begin{lstlisting}[caption={Configuraci�n de proyecto en Firebase},captionpos=b]

{
	"projects": {
		"default": "unedtfg-198720"
	}
}
\end{lstlisting} 

El entorno de demostraci�n del proyecto es accesible con conexi�n a Internet a trav�s de la siguiente URL: \url{https://unedtfg-198720.firebaseapp.com/#/}


\subsection{Detalles de implementaci�n relevantes}

Sin pretender entrar a fondo en los detalles de la implementaci�n, que puede consultarse en el repositorio de c�digo fuente del proyecto en: \url{https://github.com/predicador37/rdf-editor}, s� se ha considerado conveniente documentar determinadas decisiones:

	\begin{itemize}
	\item En el m�dulo de \textbf{modelado}, el listado de recursos se ha implementado mediante una tabla para permitir su paginado, debido a que el n�mero de clases o propiedades puede ser elevado en un modelo m�s o menos complejo. Por otra parte, los detalles de anotaciones (propiedades de clase, por ejemplo) se han implementado con listas expandibles, lo que permite una visualizaci�n compacta sin necesidad de una tabla (ya que se estima que el n�mero de estas anotaciones no ser� muy elevado).
	\item En el m�dulo de \textbf{vocabularios}, la selecci�n de estos se ha implementado como un conjunto de interruptores o\textit{ switches on/off}. Esto le permite al usuario cargar o descargar los vocabularios de forma sencilla, como si de una configuraci�n se tratase (tal es la idea).
	\item En el m�dulo de \textbf{poblamiento}, se tiene lo siguiente:
		\begin{itemize}
		\item El campo ``sujeto'' se implementa como una entrada de texto libre, puesto que lo m�s probable es que se trate de un individuo nuevo.
		\item Los campos ``predicado'' y ``objeto'' ofrecen un selector con autocompletado que permite la introducci�n de nuevos t�rminos. De esta forma, el usuario puede aprovechar los t�rminos ya introducidos en la aplicaci�n (bien por defecto bien a trav�s de otros medios) o a�adir los suyos propios, si bien ha de tener prevista su definici�n en el grafo.
	\end{itemize}
\item En el m�dulo de actividades, la importaci�n a Markdown se realiza mediante la biblioteca vue-markdown\footnote{https://github.com/miaolz123/vue-markdown}, que a su vez utiliza markdown-it\footnote{https://github.com/markdown-it/markdown-it}. Esta �ltima biblioteca implementa un procesador del formato Markdown con buen rendimiento y conforme a la especificaci�n CommonMark Spec 0.28\footnote{\url{https://spec.commonmark.org/}}, si bien a�ade extensiones de sintaxis (como tablas y tachados de Github Flavored Markdown\footnote{La sintaxis de Markdown propuesta por Github, que a�ade caracter�sticas como tablas, estilo de texto, etc. Ver \url{https://help.github.com/articles/basic-writing-and-formatting-syntax/\#styling-text}}) y otros aspectos.
	
\end{itemize}

\section{Pruebas}\cleardoublepage{}

\chapter{Resultados}

El presente proyecto ha generado los siguientes entregables:

\begin{itemize}  
	\item C�digo fuente de la aplicaci�n desarrollada en \textit{Javascript / Vue}.
	\item C�digo fuente de las bater�as de pruebas en \textit{Javascript / Jest}.
	\item C�digo fuente de la presente memoria en \LaTeX y binario en PDF.
	\item Aplicaci�n desplegada en \textit{Firebase} en \url{https://unedtfg-198720.firebaseapp.com/#/}
	\item Tablero \textit{Trello} con notas del proyecto en \url{https://trello.com/b/OXm3Sa2R/tfg-tecnolog\%C3\%ADas-sem\%C3\%A1nticas-uned} (en el momento de redacci�n de la presente memoria, de acceso privado).
\end{itemize}

El repositorio p�blico en \textit{Github} que alberga todo el c�digo fuente citado es:


\centerline{\url{https://github.com/predicador37/rdf-editor}}

En general, el producto obtenido presenta una interfaz sobria, ordenada y efectiva que permite un r�pido acceso a todas sus funcionalidades de consulta, carga de datos, modelado, etc. Es perfectamente v�lido para su trabajo con datos reales extra�dos de otros conjuntos de datos de mayor tama�o y utilizable desde el primer momento sin requerir de m�s herramientas que un sencillo navegador compatible con \textit{Javascript 2015} (cualquiera de los navegadores m�s utilizados en sus �ltimas versiones lo es). Tambi�n se ha generado, adem�s de la documentaci�n contenida en la presente memoria, un tablero \textit{Trello} con notas interesantes, recursos bibliogr�ficos, estructuraci�n de tareas a bajo nivel, etc.

En el cuadro \ref{verif-objetivos} se analiza en qu� medida y de qu� manera se ha tratado de dar cumplimiento a los objetivos marcados. 


\begin{center}
	\begin{table}
		\centering
		\begin{tabular}{p{1cm}p{7cm}p{7cm}}
			\toprule
			id & Objetivo & Respuesta \\ \midrule
			
			1 &  Facilitar el aprendizaje y la \textbf{familiarizaci�n con tecnolog�as de la Web Sem�ntica} a usuarios inexpertos. & Para verificar el cumplimiento de este objetivo ser� necesario utilizar la aplicaci�n en producci�n con actividades reales de alumnos.  \\ \midrule
			2 & Reducir al m�ximo los requisitos de uso de la herramienta para alcanzar el mayor p�blico objetivo posible. & La herramienta no requiere de instalaci�n para su uso; una vez desplegada en un servidor web, cuenta con ejemplos cargados por defecto con los que el alumno puede empezar a trabajar inmediatamente. \\ \midrule
			3 & Ofrecer a los Equipos Docentes la posibilidad de \textbf{distribuir actividades formativas} auto-contenidas para que los alumnos trabajen sobre ellas en la herramienta. & La aplicaci�n incorpora un m�dulo para cargar actividades manualmente desde texto en formato \textit{Markdown} o a trav�s de un servicio web remoto. Ver MOD-1 en \ref{sec:arquitectura-de-m�dulos} \\ \midrule
			4 & Agrupar en un solo producto \textbf{funcionalidades b�sicas de edici�n y consulta sobre grafos RDF}, de cara a cubrir las expectativas educativas en este �mbito. & El sistema desarrollado consta de varios m�dulos, entre ellos uno de modelado b�sico y otro de consultas SPARQL, que el alumno puede utilizar de forma razonablemente aut�noma. Ver MOD-4, MOD-6 en \ref{sec:arquitectura-de-m�dulos} \\ \midrule
			5 & Permitir a los alumnos comprobar \textit{in situ} los efectos de sus acciones sobre un grafo RDF y ofrecerles la \textbf{flexibilidad suficiente} como para permitirles dar rienda suelta a su creatividad e inquietudes. & El grafo por defecto que proporciona la aplicaci�n vuelve a recargarse cada vez que se refresca el navegador. El alumno es libre de realizar las operaciones que desee, incluso de �ndole destructiva, y el estado inicial de la aplicaci�n se restablecer� f�cilmente. Ver MOD-4, MOD-5 en \ref{sec:arquitectura-de-m�dulos} \\ \midrule
			6 & Obtener un MVP que pueda ser evolucionado y poner su c�digo a disposici�n de la comunidad educativa para revertir sobre ella su valor aportado. & El propio sistema es un MVP preparado para ser evolucionado con distintas caracter�sticas y mejoras sobre las desarrolladas inicialmente. \\ 
			\bottomrule
		\end{tabular}
		\caption{Verificaci�n del cumplimiento de los objetivos iniciales.}\label{verif-objetivos}
	\end{table}
\end{center}

\section{Accesibilidad}

La accesibilidad web tiene como objetivo extender el uso de la web al mayor n�mero posible de personas, incluyendo a aquellos con cualquier tipo de limitaci�n, ya sea geogr�fica, socio-econ�mica, funcional, etc.

Actualmente, la legislaci�n espa�ola (\textit{Real Decreto 1494/2007 de 12 de noviembre, por el que se aprueba el Reglamento sobre las condiciones b�sicas para el acceso de las personas con discapacidad a las tecnolog�as, productos y servicios relacionados con la sociedad de la informaci�n y medios de comunicaci�n social}\footnote{\url{https://www.boe.es/buscar/act.php?id=BOE-A-2007-19968}}) establece que la\textit{``Norma UNE 139803:2012: Requisitos de accesibilidad para contenidos en la Web''} define los requisitos de accesibilidad para los contenidos web. Esta norma es equivalente a los criterios WCAG 2.0\footnote{Web Content Accesibility Guidelines 2.0, \url{https://www.w3.org/TR/WCAG20/}}.

Para verificar el grado de accesibilidad de la aplicaci�n se ha utilizado el complemento Axe\footnote{\url{https://www.deque.com/axe/}} para el navegador Firefox.Es una de las herramientas m�s utilizadas por la comunidad de desarrolladores, abierta y flexible, que permite validar si el HTML generado es conforme a las pautas WCAG 2.0 (y, por tanto, a la norma obligada en Espa�a).

Validar la accesibilidad de una aplicaci�n 100\% \textit{Javascript} es complejo, dado que su contenido y el �rbol del documento (DOM\footnote{Document Object Model, una API para documentos HTML y XML. Ver \url{https://www.w3.org/TR/WD-DOM/introduction.html}}) se generan din�micamente. No obstante, el funcionamiento a modo de complemento de navegador de la herramienta utilizada ha permitido facilitar esta tarea y su aplicaci�n en los distintos m�dulos de la aplicaci�n (ver ejemplo en figura \ref{fig:acc01})

\begin{figure}[h]
	\centering
	\includegraphics[scale=0.25]{axe}
	\caption{Validaci�n de accesibilidad con \textit{Axe} para \textit{Mozilla Firefox}.}\label{fig:acc01}
\end{figure}

Otra herramienta com�nmente utilizada para la validaci�n de la accesibilidad en aplicaciones web es \textit{Lighthouse}\footnote{\url{https://developers.google.com/web/tools/lighthouse/\#devtools}}, instalable como un plugin del navegador \textit{Chrome}. Seg�n su esquema de reglas, \textit{RDFplay} es 100\% accesible, tal y como se puede comprobar en la figura \ref{fig:acc02}

\begin{figure}[h]
	\centering
	\includegraphics[scale=0.25]{lighthouse}
	\caption{Validaci�n de accesibilidad con \textit{Axe} para \textit{Mozilla Firefox}.}\label{fig:acc02}
\end{figure}

Lighthouse proporciona otros indicadores que no son relevantes aqu�, puesto que no era el prop�sito de este proyecto desarrollar la aplicaci�n resultante como una PWA (\textit{Progressive Web Application}). En cuanto al pobre rendimiento mostrado en el indicador correspondiente, este depende de muchos factores; el factor principal es el alojamiento de la aplicaci�n en una plataforma gratuita y el elevado tama�o a descargar debido a las bibliotecas de \textit{Comunica}, que han de incluirse en su totalidad mientras se encuentren en versiones inestables.

\section{Visualizaci�n en dispositivos m�viles}

La biblioteca de componentes utilizada, \textit{Vuetify}, es \textit{``mobile friendly''}, lo que facilita el desarrollo de una aplicaci�n visualizable en dispositivos m�viles. Sin embargo, no basta tan solo con utilizar la biblioteca; determinados componentes como listas tienen implementadas caracter�sticas de elipsis de contenidos que hacen que su visualizaci�n en un dispositivo m�vil no sea la m�s adecuada.

Dado que durante todo el proceso de desarrollo la aplicaci�n ha estado desplegada en \textit{Firebase} y por tanto accesible desde Internet, se ha podido revisar y mejorar la visualizaci�n en dispositivos m�viles de forma incremental e iterativa. El resultado final consiste en una aplicaci�n perfectamente visualizable en un dispositivo m�vil, tal y como se puede comprobar manualmente o a trav�s de herramientas de validaci�n como \textit{Google Mobile Friendly Test}\footnote{\url{https://search.google.com/test/mobile-friendly}} (ver figura \ref{fig:mobile01}).

\begin{figure}[h]
	\centering
	\includegraphics[scale=0.25]{mobile_friendly}
	\caption{Resultados de \textit{Google Mobile Friendly Test}.}\label{fig:mobile01}
\end{figure}

\section{Rendimiento}



\section{Calidad}

Como medida fundamental de calidad del c�digo fuente se ha elegido la complejidad ciclom�tica, un indicador que arroja datos sobre la futura facilidad de mantenimiento del mismo.

La biblioteca \textit{Eslint}, que realiza an�lisis est�tico de c�digo en cada compilaci�n y es utilizada en el proyecto, permite configurar distintas reglas. Adem�s de las incorporadas por defecto\footnote{Ver reglas por defecto en \url{https://eslint.org/docs/rules/}}, se ha configurado la regla de complejidad con un valor m�ximo de 10. Como se puede comprobar en la figura \ref{fig:eslint01}), no hay ninguna regla invalidante a la hora de compilar y ejecutar la aplicaci�n en el momento de finalizaci�n de su desarrollo.

\begin{figure}[h]
	\centering
	\includegraphics[scale=0.25]{eslint}
	\caption{Compilaci�n y an�lisis est�tico con \textit{Eslint}}.\label{fig:eslint01}
\end{figure}

\section{Rendimiento}

Es importante destacar que la aplicaci�n desarrollada, al ejecutarse completamente en el navegador del usuario, puede presentar importantes limitaciones de rendimiento debido fundamentalmente al uso de memoria RAM.

Conjuntos de datos del orden de \textit{Kilobytes} son manejables, pero las pruebas de carga de un fichero TTL de 22 \textit{Megabytes} revelan un consumo final de RAM de en torno a 500 MB, as� como la \textbf{inviabilidad de lanzar consultas sin limitar los resultados a dicho grafo}.

\begin{figure}[h]
	\centering
	\includegraphics[width = 500pt]{rendimiento_import_22}
	\caption{Memoria utilizada para un conjunto de datos de 22 MB.}
\end{figure}

Sin embargo, dado que el objetivo del presente proyecto no es competir con bases de datos RDF y \textit{backends} SPARQL que contengan millones de tripletas sino facilitar el aprendizaje sobre grafos sencillos, estas limitaciones no se consideran problem�ticas siempre y cuando se tengan en cuenta a la hora de trabajar con distintos conjuntos de datos.\cleardoublepage{}

\chapter{Conclusiones y l�neas de trabajo futuras}


\section{Conclusiones}

Alimentar la Web con datos estructurados y contextualizados con metadatos no ha de ser una labor exclusiva de entornos acad�micos y universitarios. Si bien son estas organizaciones las que han modelado dominios de informaci�n y liberado conjuntos de datos de forma masiva a lo largo de los �ltimos a�os, es necesario extender esta responsabilidad a otros organismos p�blicos y privados y, por qu� no, a potenciales usuarios o clientes particulares. Solo a partir de ese momento, el relato inicial de Berners-Lee\cite{bernerslee01} en su primera definici�n de la Web Sem�ntica ser� una realidad en nuestra Sociedad.

No obstante, a�n queda mucho trabajo por hacer. Organizaciones p�blicas y privadas han de concienciarse de la enorme importancia de poner a disposici�n de la Sociedad en general datos estructurados en formatos reutilizables. S�lo de esta forma crecer� el inter�s en su tratamiento y explotaci�n, as� como en el desarrollo de agentes inteligentes que sean capaz de procesarlos para generar nuevos resultados de an�lisis. No hay m�s que recurrir a los datos del Estudio de Caracterizaci�n del Sector Infomediario en Espa�a\cite{ontsi01} proporcionados por el Observatorio Nacional de las Telecomunicaciones y de la Sociedad de la Informaci�n, que recogen que el 70\% de las empresas encuestadas accede a datos no estructurados, frente al 69\% que lo hace a datos en formatos abiertos (si bien no se especifica si se trata de datos a nivel de datos abiertos enlazados o \textit{linked open data}).

La comunidad de tecnol�gos y desarrolladores de aplicaciones para la Web Sem�ntica debe trabajar por la democratizaci�n de esta tecnolog�a. Con los niveles de complejidad conceptual y de uso actuales, su difusi�n entre el p�blico general se hace complicada. En este sentido, las grandes revoluciones surgidas en torno al mundo del desarrollo de \textit{frontend} web en general (y \textit{Javascript} en particular) y el nivel de madurez alcanzado en estos momentos por las tecnolog�as de la Web Sem�ntica, favorecen la llegada del momento m�s propicio en los �ltimos veinte a�os para lograr un impulso real. A�n son pocas las comunidades de desarrolladores trabajando en ofrecer servicios sem�nticos a trav�s de \textit{Javascript}, pero la existencia de un \textit{``community group''}\cite{rdfw3ccg} que aglutina esfuerzos para aunar a dichas comunidades en un s�lo proyecto es una noticia prometedora de cara al futuro de esta tecnolog�a.

Como peque�a contribuci�n a estos esfuerzos de difusi�n y democratizaci�n, se ha desarrollado el presente proyecto, que representa la culminaci�n de meses de trabajo pero tambi�n el inicio de la vida de un producto. Se sientan las bases para obtener una herramienta acad�mica que permita acercar las tecnolog�as de la Web Sem�ntica a cualquier no iniciado. Si bien en esta primera versi�n ya se cuenta con una aplicaci�n funcional, el estado de las bibliotecas que integra y las posibilidades de RDF y SPARQL sugieren unas expectativas de evoluci�n lo suficientemente optimistas como para continuar su desarrollo y no abandonar su mantenimiento.

A partir de este punto, la hoja de ruta que se propone para el producto desarrollado en el presente proyecto es clara: en un primer curso acad�mico, comenzar a utilizarlo a modo de prueba piloto en asignaturas t�cnicas relacionadas con la Web Sem�ntica a trav�s de actividades opcionales (para mejorar la nota total del alumno, por ejemplo) e ir evaluando su idoneidad de cara a su manejo en el entorno para el que fue pensado. Paralelamente, la aplicaci�n puede seguir siendo mejorada y perfeccionada, en caso de aparici�n de defectos inesperados (algo, por otra parte, perfectamente asumible hoy en d�a).

Un claro compromiso por parte del autor del proyecto debe ser para con la Comunidad Educativa, y especialmente para los grupos de trabajo como el \textit{RDF Javascript Libraries Community Group} por su inestimable ayuda prestada. Por tanto, parece razonable trabajar en una futura traducci�n al idioma ingl�s para permitir su publicaci�n como repositorio de c�digo abierto y revertir el desarrollo a la Comunidad.

No se podr�a concluir este cap�tulo de conclusiones sin valorar las competencias tecnol�gicas y organizativas que el desarrollo del proyecto ha permitido adquirir a su autor: profundizaci�n en el modelado RDF, las arquitecturas de aplicaciones de la Web Sem�ntica, los motores de consultas SPARQL y el desarrollo en marcos de trabajo \textit{Javascript} completos y bien estructurados como Vue.js (una inversi�n, esta �ltima, m�s que rentable en t�rminos de utilizaci�n de la tecnolog�a).

Tomar decisiones entre altas dosis de incertidumbre no es sencillo, y menos a�n cuando la oferta tecnol�gica es numerosa y variada (como es el caso, por ejemplo, de las bibliotecas de componentes gr�ficos para Vue). Adem�s, la complejidad inicial del desarrollo en \textit{Javascript} no presenta una curva de aprendizaje suave, especialmente en la configuraci�n del entorno necesario: empaquetadores, \textit{``transpilers''} con sus m�dulos asociados, existencia de diversos est�ndares conviviendo entre distintos m�dulos que complican su integraci�n, etc.

Por otra parte, los trabajos previos de implementaci�n de RDF en \textit{Javascript} tambi�n son numerosos y confusos (prueba de ello es la reciente noticia de Junio de este mismo a�o, 2018, indicando la unificaci�n de las diferentes implementaciones existentes en una �nica implementaci�n). Afortunadamente, la ayuda y los consejos de los propios desarrolladores, ya con experiencia, ha permitido enfocar el proyecto con las estrategias adecuadas.

En resumidas cuentas, se ha tratado de obtener un producto apto para la formaci�n acad�mica en tecnolog�as de la Web Sem�ntica, cumpliendo razonablemente con los objetivos iniciales y abriendo un camino al desarrollo de una herramienta mucho m�s completa que pueda aprovecharse de las tecnolog�as emergentes en el �rea del \textit{frontend} web. Se han utilizado bibliotecas y m�dulos est�ndares dentro de la situaci�n actual, lo que permitir� una mayor facilidad de mantenimiento en el futuro y pretende abrir el camino a evoluciones m�s ambiciosas. A fecha de finalizaci�n de la presente memoria, ya se est�n manteniendo conversaciones con el Director del Proyecto para el futuro uso de la aplicaci�n en la actividad docente, lo que permitir�, sin duda, obtener \textit{feedback} de cara a su futura mejora. El tablero Kanban del proyecto y el propio repositorio en Github est�n disponibles desde este momento para incorporar nuevas tarjetas y peticiones de evoluci�n o mejora.

\section{L�neas de trabajo futuras}

Como punto de partida para futuras mejoras y nuevas funcionalidades, se proponen las siguientes l�neas de trabajo:

\begin{itemize}
\item Integrar la primera versi�n estable (una vez publicada) de los distintos m�dulos de la plataforma Comunica y optimizar el uso de dependencias.
\item Integrar las futuras versiones del marco de trabajo de componentes Vuetify y utilizar nuevos componentes como la estructura jer�rquica (en �rbol) para representar clases y propiedades.
\item Integrar o desarrollar consultas SPARQL con Comunica a grafos en RDFXML\footnote{\url{https://github.com/comunica/comunica/issues/140}}
\item Incrementar la cobertura de pruebas unitarias y extremo a extremo
\item Re-orientar las operaciones de consulta a trav�s del m�dulo \textit{actor-init-sparql-rdfjs} (ver secci�n \ref{sec:motor-de-consultas-en-el-proyecto})
\item Funcionalidad de exportar a un formato JSON-raw (una lista de objetos con ternas).
\item Posibilidad de modificar los prefijos a incluir en las consultas. En general, facilitar el uso de prefijos.
\item Enriquecer y refactorizar la funcionalidad de modelado.
\item Soportar m�ltiples grafos.
\item Revisar la carga de t�rminos b�sicos y estudiar la viabilidad de carga en l�nea desde una ontolog�a est�ndar.
\item Traducci�n de la aplicaci�n al idioma ingl�s y puesta a disposici�n de las distintas comunidades de desarrolladores y usuarios.
\end{itemize} 




\cleardoublepage{}


\raggedright
\fancyhead[LO]{\leftmark}
\bibliographystyle{unsrt}
\bibliography{biblio/biblio-pfg}


\printglossary[type=\acronymtype,title=Acr�nimos]
\cleardoublepage{}

\appendix
\cleardoublepage{}\renewcommand{\appendixname}{Anexo}

\chapter{<T�tulo Anexo A>}

<...>

\section{<Primera secci�n anexo>}

\subsection{\label{subsec:<Primera-subsecci=0000F3n-anexo>}<Primera subsecci�n
anexo>}



\end{document}
