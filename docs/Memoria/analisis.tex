\chapter{An�lisis}

\section{Captura y documentaci�n de requisitos}

\subsection{Captura de requisitos}

La t�cnica de captura de requisitos utilizada para este proyecto ha sido, fundamentalmente, \textbf{la entrevista} con el tutor. Se eligi� esta t�cnica por los siguientes motivos:

\begin{itemize}  
	\item Las entrevistas, bien a distancia o presenciales (y especialmente estas �ltimas), permiten una mayor implicaci�n del usuario en la captura de requisitos.
	\item Combinada con una maqueta o prueba de concepto, una entrevista presencial puede dar lugar a la aparici�n de nuevos requisitos de producto, cambios en las especificaciones e incluso en el enfoque y objetivos del mismo.
	\item Permite la pr�ctica de la escucha activa y la sugerencia de ideas por parte del analista, aportando un valor a�adido que enriquece la simple captura de requisitos.
	\item Es claramente la t�cnica m�s obvia, directa y accesible en el contexto de la realizaci�n del proyecto.
\end{itemize}

Concretamente, se han llevado a cabo varias entrevistas utilizando la plataforma colaborativa Skype y una presencial, en la que el autor de este proyecto se ha desplazado a la sede del departamento en Madrid con objeto de conseguir una comunicaci�n m�s fluida y un mayor entendimiento a la hora de consensuar las necesidades y funcionalidades requeridas del producto.

La primera entrevista a distancia propici� un intercambio de documentos e ideas que desemboc� en la elaboraci�n del documento del anteproyecto. 
\vspace*{\baselineskip}
\begin{center}
\begin{table}[htb]
\centering
\begin{tabular}{p{1cm}p{2cm}p{10cm}}
	\toprule
	id & Fecha & Resumen de la entrevista \\
	\toprule
	1 & 18/10/17 & Primer contacto y comunicaci�n de ideas iniciales para la confecci�n del anteproyecto  \\ \midrule
	2 & 21/02/17 & Consolidaci�n de ideas y aportaci�n de m�s documentaci�n (v�deos sobre prototipos de cuaderno, documentos de texto con descripciones, etc.)  \\
	3 & 28/03/18 & Primera demo a modo de POC con un entorno capaz de a�adir tripletas.  \\ \midrule
	4 & 10/07/18 & Reuni�n presencial con demostraci�n \textit{in-situ} de los m�dulos de modelado e importaci�n/exportaci�n. Tiene lugar una tormenta de ideas y se enfoca el proyecto de otro modo, modificando sus objetivos hacia una herramienta formativa..  \\ \midrule
	5 & 4/08/18 & Revisi�n de los �ltimos avances con la integraci�n de un endpoint SPARQL en el frontend y planificaci�n del resto de funcionalidade requeridas.  \\ 
	\bottomrule
\end{tabular}
\caption{Sesiones de entrevistas}
\label{tab-1}
\end{table}
\end{center}

\subsection{Documentaci�n}

Para documentar la captura de requisitos, se utilizar� la t�cnica de casos de uso. Se descarta la incorporaci�n de diagramas UML de casos de uso 

\section{Necesidades}


\section{Casos de uso}

\subsection{Caso de uso 1}

\subsubsection{Contexto}

\subsubsection{�mbito}

\subsubsection{Nivel}

\subsubsection{Actor principal}

\subsubsection{Participantes e interesados}

\subsubsection{Precondiciones}

\subsubsection{Garant�as m�nimas}

\subsubsection{Garant�as de �xito}

\subsubsection{Disparador}

\subsubsection{Descripci�n}

\subsubsection{Extensiones}