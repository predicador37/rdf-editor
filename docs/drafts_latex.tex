\begin{table}[]
	\begin{tabularx}{8cm}{|X|X|}
		\toprule
		CU \#                     & \multicolumn{2}{l}{Lanzar consulta SPARQL al grafo local}                   \\ \midrule
		Contexto                  & \multicolumn{2}{l}{Este caso de uso permite a un estudiante lanzar consultas SPARQL más o 
		menos complejas al grafo local a través de una entrada de texto sencilla. El sistema presentará una consulta por defecto con una serie de prefijos de uso común precargados. } \\ \midrule
		Ámbito                    & \multicolumn{2}{l}{Módulo de consultas}          \\ \midrule
		Nivel                     & \multicolumn{2}{l}{tarea principal}                     \\ \midrule
		Actor principal           & \multicolumn{2}{l}{Estudiante}                        \\ \midrule
		Participantes e intereses & Participante  & Interés                                       \\ \midrule
		& nombre                                  & interés                                       \\ \midrule
		& nombre                                  & interés                                       \\ \midrule
	\end{tabularx}
	\begin{tabularx}{8cm}{|X|X|}
		Precondiciones            & \multicolumn{2}{l}{lo que ya se espera que sea el estado del mundo}                     \\ \midrule
		Garantías mínimas         & \multicolumn{2}{l}{intereses como protegidos de cualquier salida}                       \\ \midrule
		Garantías de éxito        & \multicolumn{2}{l}{los intereses como satisfechos en un final con éxito}                \\ \midrule
		Disparador                & \multicolumn{2}{l}{la acción sobre el sistema mediante la que comienza el caso de uso}  \\ \midrule
		Descripción               & Paso                                    & Acción                                        \\ \midrule
		& 1                & pasos desde el escenario con el disparador al objetivo de entrega    \\ \midrule
		Extensiones               & Paso                                    & Acción alternativa                            \\ \midrule
		&                                         &                                               \\ \bottomrule
	\end{tabularx}
	\caption{Caso de uso 1}
	\label{cu-1}
\end{table}





\vspace*{\baselineskip}
\begin{center}
	\begin{table}[htb]
		\centering
		\begin{tabular}{p{1cm}p{15cm}}
		
			\addlinespace[0.5em]
			id & Necesidad  \\
			\toprule
			\addlinespace[1em]
			1 &  Permitir a usuarios de la UNED  de distintos colectivos etiquetar y generar sus propios cuadernos con información y metainformación semántica. \\ \midrule
			\addlinespace[1em]
			2 &  Permitir a dichos usuarios realizar consultas sobre sus cuadernos  \\ \midrule
			\addlinespace[1em]
			3 &  Desarrollar una interfaz web que permita al usuario gestionar tripletas RDF.  \\ \midrule
			\addlinespace[1em]
			4 &  Desarrollar un módulo de generación de consultas SPARQL a partir de consultas de lectura y escritura en formato JSON.  \\ \midrule
			\addlinespace[1em]
			5 &  Desarrollar los correspondientes productos de interés para el usuario: consultas exportadas en forma diversa (CSV, JSON) o embebidas en una plantilla HTML significativa para el usuario. \\ \midrule
			\addlinespace[1em]
			6 &  Ofrecer la posibilidad de variar interfaces de entrada y exportadores en función de los distintos colectivos de usuario utilizando los metadatos sobre cuadernos RDF previamente almacenados en una base de datos relacional. \\ \midrule
			\addlinespace[1em]
			7 &  Permitir mostrar en pantalla una serie de términos como punto de partida que el usuario pueda utilizar para construir ternas RDF y relaciones entre ellas. \\ \midrule
			\addlinespace[1em]
			8 &  Permitir mostrar en pantalla una serie de términos como punto de partida que el usuario pueda utilizar para construir ternas RDF y relaciones entre ellas. \\ \midrule
			\addlinespace[1em]
			9 &  Ofrecer al usuario una visualización sencilla y correcta de su modelo que proporciona una perspectiva adecuada sobre la que trabajar. \\ \midrule
			\addlinespace[1em]
			10 &  Presentar una interfaz de mantenimiento del grafo: vocabulario e instancias (conceptualización y poblamiento de una ontología). \\ \midrule
			\addlinespace[1em]
			11 &  Importar y exportar información estructurada en formatos semánticos estándar. \\ \midrule
			\addlinespace[1em]
			12 &  Extender con vocabularios tales como SKOS y OWL. \\ 
			\bottomrule
			\addlinespace[1em]
		\end{tabular}
		\caption{Necesidades iniciales}
		\label{tab-1}
	\end{table}
\end{center}

